%!TEX root = ends.tex
%
%
%
%
%
%
%
%
%
%
%%%%%%%%%%%%%%%%%%%%%%%%%%%%%%%%%%%%%%%%%%%%%%%%%%%%%%%%%%%%%%%%%%%%%%%%%%%
%%%%%%%%%%%%%%%%%%%%%%%%%%%%%%%%%%%%%%%%%%%%%%%%%%%%%%%%%%%%%%%%%%%%%%%%%%%
%%%%%%%%%%%%%%%%%    Section : Definitions and examples    %%%%%%%%%%%%%%%%
%%%%%%%%%%%%%%%%%%%%%%%%%%%%%%%%%%%%%%%%%%%%%%%%%%%%%%%%%%%%%%%%%%%%%%%%%%%
%%%%%%%%%%%%%%%%%%%%%%%%%%%%%%%%%%%%%%%%%%%%%%%%%%%%%%%%%%%%%%%%%%%%%%%%%%%
\section{Definitions and examples}


This section contains all the definitions, some standard but useful preliminary results as well as some examples.
% and can be found in any good book about geometric group theory, see for example~\cite{DelaHarpe2000}.\todo{Peut-être retravailler l'intro.}
%
%
%
%
%
%
%
%
%
%
%%%%%%%%%%%%%%%%%%%%%%%%%%%%%%%%%%%%%%%%%%%%%%%%%%%%%%%%%%%%%%%%%%%%%%%%%%%
%%%%%%%%%%%%%%%%%%%%%%%%%%%%%%%%%%%%%%%%%%%%%%%%%%%%%%%%%%%%%%%%%%%%%%%%%%%
%%%%%%%%    Subsection : Ends of Schreier graphs and property FW    %%%%%%%
%%%%%%%%%%%%%%%%%%%%%%%%%%%%%%%%%%%%%%%%%%%%%%%%%%%%%%%%%%%%%%%%%%%%%%%%%%%
%%%%%%%%%%%%%%%%%%%%%%%%%%%%%%%%%%%%%%%%%%%%%%%%%%%%%%%%%%%%%%%%%%%%%%%%%%%
\subsection{Ends of Schreier graphs and property FW}
\label{Subsection:FW}
%
%
%
%
%
In what follows, we will always assume that generating sets of groups are \emph{symmetric}, that is we will look at $S\subset G$ such that $s\in S$ if and only if $s^{-1}\in S$.
Our graphs will be undirected and we will sometimes identify a graph with its vertex set.
%
%
\begin{defn}
Let $G$ be a group, $H$ a subgroup of $G$ and $S$ a symmetric generating set. The \emph{(left) Schreier graph} $\Sch(G,H;S)$ is the graph with vertices the left cosets $gH=\setst{gh}{h\in H}$ and for every $\{s,s^{-1}\}\subset S$ an edge between $gH$ and $sgH$.% for every $s$ in $S$.
\end{defn}
%
%
Observe that with the above definition, a Schreier graph may have loops as well as multiple edges. Moreover, in $\Sch(G,H;S)$ every vertex has degree $\abs S$.

If $X$ is a set with a left action $G\curvearrowright X$ and $S$ is a symmetric generating set for $G$, the corresponding (left) \emph{orbital graph} $\Sch_{\mathcal O}(G,X;S)$ is the graph with vertex set $X$ and with an edge between $x$ and $y$ for every $s$ in $S$ such that $s.x=y$.
As the notation suggests, these two definitions are two faces of the same coin. Indeed, we have $\Sch(G,H;S)=\Sch_{\mathcal O}(G,G/H;S)$ for the  action by left multiplication of $G$ on $G/H$, while   for every $x\in X$, the graph $\Sch(G,\stab_G(x);S)$ is equal to the connected component of $x$ in $\Sch_{\mathcal O}(G,X;S)$.

Schreier graphs are generalizations of the well-known \emph{Cayley graphs}, with $\Cayley(G;S)=\Sch(G,\{1\};S)$, see Figures \ref{Figure:CayleyOfZ} and \ref{Figure:CayleyOfZ2} for some examples.

If $S$ and $T$ are two generating sets of $G$, the graphs $\Sch(G,H;S)$ and $\Sch(G,H;T)$ does not need to be isomorphic. However, if both $S$ and $T$ are finite, then $\Sch(G,H;S)$ and $\Sch(G,H;T)$ are quasi-isometric, see \cite[IV.B.21.iii]{DelaHarpe2000} for a proof and Figure~\ref{Figure:CayleyOfZ} for an example. The only fact we will use about quasi-isometries is that they preserve ``large-scale properties'' of the graph, as for example the number of ends.
Observe that the requirement that both $S$ and $T$ are finite is crucial for the existence of a quasi-isometry between the corresponding Schreier graphs. Indeed, for every group  $\Cayley(G;G\setminus\{1\})=\Sch(G,\{1\};G\setminus\{1\})$ is always a complete graph on $\abs{G}$ vertices, in particular  $\Cayley(\Z;\{1\})$ and $\Cayley(\Z;\Z\setminus\{0\})$ are not quasi-isometric.

%
%
\begin{figure}[htbp]\centering
\begin{tikzpicture}[node distance=1.14cm,every node/.style=blacknode]
\useasboundingbox (-3.1,-1.55) rectangle (7.62,0.08);
\node (0){} ;


\node (1)[right of=0]{} ;
\node (2)[right of=1]{} ;
\node (3)[right of=2]{} ;
\node (4)[right of=3]{} ;
\coordinate[right =0.5cm of 4] (b) ;
\graph[edge={thick}]{(a)--[dotted](0)--(1)--(2)--(3)--(4)--[dotted](b)} ;

\begin{scope}[yshift=-1cm]
\node (0){} ;
%\coordinate[left =0.5cm of 0] (a) ;
\coordinate[left =1cm of 0] (-1) ;
\coordinate[left =2cm of 0] (-2) ;
\coordinate[left =3cm of 0] (-3) ;
\node (1)[right of=0]{} ;
\node (2)[right of=1]{} ;
\node (3)[right of=2]{} ;
\node (4)[right of=3]{} ;
%\coordinate[right =0.5cm of 4] (b) ;
\coordinate[right =1cm of 4] (5) ;
\coordinate[right =2cm of 4] (6) ;
\coordinate[right =3cm of 4] (7) ;
\graph[edge={thick}]{(-2)--[bend left,red,dotted](0)--[bend left,red](2)--[bend left,red](4)--[bend left,red,dotted](6), (-1)--[bend left,red,dotted](1)--[bend left,red](3)--[bend left,red,dotted](5) } ;
\graph[edge={thick}]{(-3)--[bend right,blue,dotted](0)--[bend right,blue](3)--[bend right,blue,dotted](6), (-2)--[bend right,blue,dotted](1)--[bend right,blue](4)--[bend right,blue,dotted](7),(-1)--[bend right,blue,dotted](2)--[bend right,blue,dotted](5)} ;
\end{scope}
\end{tikzpicture}
\caption{Fragments of two Cayley graphs of $\Z$ ($2$ ends), for the standard generating set $\{\pm1\}$ and for the generating set $\{\textcolor{red}{\pm2},\textcolor{blue}{\pm3}\}$.}
\label{Figure:CayleyOfZ}
\end{figure}
\begin{figure}[htbp]\centering
\begin{subfigure}{0.5\textwidth}
\centering
\begin{tikzpicture}
%\foreach \i in {0,...,4}
%{
%	\draw[thick, blue,dotted] (0.5,-\i)--(1,-\i) ;
%	\draw[thick, blue,dotted] (0.5,-\i)--(1,-\i) ;	
%}
\foreach \i in {1,...,25}
{
        \pgfmathtruncatemacro{\y}{(\i - 1) / 5} ;
        \pgfmathtruncatemacro{\x}{\i - 5 * \y} ;
        \pgfmathtruncatemacro{\label}{\x + 5 * (4 - \y)} ;
        \node[blacknode] (\label) at (1*\x,-1*\y) {} ;
}
\foreach \i in {1,...,5}{
	\draw[thick, blue,dotted] (0.5,-\i+1)--(0.95,-\i+1) ;
	\draw[thick, blue,dotted] (5.5,-\i+1)--(5.05,-\i+1) ;
	\draw[thick, red,dotted] (\i,-4.5)--(\i,-4.05) ;
	\draw[thick, red,dotted] (\i,0.5)--(\i,0.05) ;
	\foreach \j in {0,...,3}{
		\pgfmathtruncatemacro{\x}{1+5*(\i-1)+\j} ;
		\pgfmathtruncatemacro{\y}{\x+1} ;
		\draw[thick, blue](\x)--(\y) ;
		\pgfmathtruncatemacro{\zz}{\i+5*\j} ;
		\pgfmathtruncatemacro{\t}{\zz+5} ;
		\draw[thick, red] (\zz)--(\t) ;
	}
}
\end{tikzpicture}

%\caption{Lorem ipsum}
\end{subfigure}%
\begin{subfigure}{0.5\textwidth}
\centering
\scalebox{0.8}{\begin{tikzpicture}[every node/.style=blacknode]
\useasboundingbox (-2.9,-2.9) rectangle (2.9,2.9);
\rotatebox{45}{
\node (0){};
\foreach \i/\j in {E/right,W/left}{
\node (\i)[\j = 2cm of 0]{} ;
\node (\i\i)[\j = 1cm of \i]{} ;
	\node (\i\i N)[above = 0.5cm of \i\i]{} ;
		\coordinate[above =0.18cm of \i\i N] (\i\i NN) ;
		\coordinate[right =0.18cm of \i\i N] (\i\i NE) ;
		\coordinate[left =0.18cm of \i\i N] (\i\i NW) ;
	\node (\i\i S)[below = 0.5cm of \i\i]{} ;
		\coordinate[below =0.18cm of \i\i S] (\i\i SS) ;
		\coordinate[right =0.18cm of \i\i S] (\i\i SE) ;
		\coordinate[left =0.18cm of \i\i S] (\i\i SW) ;
	\node (\i\i\i)[\j = 0.5cm of \i\i]{} ;
		\coordinate[\j =0.18cm of \i\i\i] (\i\i\i\i) ;
		\coordinate[above =0.18cm of \i\i\i] (\i\i\i N) ;
		\coordinate[below =0.18cm of \i\i\i] (\i\i\i S) ;
\node (\i N)[above = 1cm of \i ]{} ;
	\node (\i NN)[above = 0.5cm of \i N]{} ;
		\coordinate[above =0.18cm of \i NN] (\i NNN) ;
		\coordinate[right =0.18cm of \i NN] (\i NNE) ;
		\coordinate[left =0.18cm of \i NN] (\i NNW) ;
	\node (\i NW)[left = 0.5cm of \i N]{} ;
		\coordinate[above =0.18cm of \i NW] (\i NWN) ;
		\coordinate[below =0.18cm of \i NW] (\i NWS) ;
		\coordinate[left =0.18cm of \i NW] (\i NWW) ;
	\node (\i NE)[right = 0.5cm of \i N]{} ;
		\coordinate[above =0.18cm of \i NE] (\i NEN) ;
		\coordinate[below =0.18cm of \i NE] (\i NES) ;
		\coordinate[right =0.18cm of \i NE] (\i NEE) ;
\node (\i S)[below = 1cm of \i ]{} ;
	\node (\i SS)[below = 0.5cm of \i S]{} ;
		\coordinate[below =0.18cm of \i SS] (\i SSS) ;
		\coordinate[right =0.18cm of \i SS] (\i SSE) ;
		\coordinate[left =0.18cm of \i SS] (\i SSW) ;
	\node (\i SW)[left = 0.5cm of \i S]{} ;
		\coordinate[above =0.18cm of \i SW] (\i SWN) ;
		\coordinate[below =0.18cm of \i SW] (\i SWS) ;
		\coordinate[left =0.18cm of \i SW] (\i SWW) ;
	\node (\i SE)[right = 0.5cm of \i S]{} ;
		\coordinate[above =0.18cm of \i SE] (\i SEN) ;
		\coordinate[below =0.18cm of \i SE] (\i SES) ;
		\coordinate[right =0.18cm of \i SE] (\i SEE) ;
\graph[edge={thick,color=blue}]{
(0)--(\i )--(\i\i)--(\i\i\i)--[densely dotted](\i\i\i\i),
(\i SWW)--[densely dotted](\i SW)--(\i S)--(\i SE)--[densely dotted](\i SEE),
(\i NWW)--[densely dotted](\i NW)--(\i N)--(\i NE)--[densely dotted](\i NEE),
(\i NNW)--[densely dotted](\i NN)--[densely dotted](\i NNE),
(\i SSW)--[densely dotted](\i SS)--[densely dotted](\i SSE),
(\i\i NW)--[densely dotted](\i\i N)--[densely dotted](\i\i NE),
(\i\i SW)--[densely dotted](\i\i S)--[densely dotted](\i\i SE)
};
\graph[edge={thick,color=red}]{
	(\i SSS)--[densely dotted](\i SS)--(\i S)--(\i)--(\i N)--(\i NN)--[densely dotted](\i NNN),
	(\i\i SS)--[densely dotted](\i\i S)--(\i\i)--(\i\i N)--[densely dotted](\i\i NN),
	(\i\i\i N)--[densely dotted](\i\i\i)--[densely dotted](\i\i\i S),
	(\i NES)--[densely dotted](\i NE)--[densely dotted](\i NEN),
	(\i SES)--[densely dotted](\i SE)--[densely dotted](\i SEN),
	(\i NWS)--[densely dotted](\i NW)--[densely dotted](\i NWN),
	(\i SWS)--[densely dotted](\i SW)--[densely dotted](\i SWN)
};
}
\foreach \i/\j in {N/above,S/below}{
\node (\i)[\j = 2cm of 0]{} ;
\node (\i\i)[\j = 1cm of \i]{} ;
	\node (\i\i W)[left = 0.5cm of \i\i]{} ;
		\coordinate[above =0.18cm of \i\i W] (\i\i WN) ;
		\coordinate[below =0.18cm of \i\i W] (\i\i WS) ;
		\coordinate[left =0.18cm of \i\i W] (\i\i WW) ;	
	\node (\i\i E)[right = 0.5cm of \i\i]{} ;
		\coordinate[above =0.18cm of \i\i E] (\i\i EN) ;
		\coordinate[below =0.18cm of \i\i E] (\i\i ES) ;
		\coordinate[right =0.18cm of \i\i E] (\i\i EE) ;
	\node (\i\i\i)[\j = 0.5cm of \i\i]{} ;
		\coordinate[\j =0.18cm of \i\i\i] (\i\i\i\i) ;
		\coordinate[right =0.18cm of \i\i\i] (\i\i\i E) ;
		\coordinate[left =0.18cm of \i\i\i] (\i\i\i W) ;
\node (\i E)[right = 1cm of \i ]{} ;
	\node (\i EE)[right = 0.5cm of \i E]{} ;
		\coordinate[above =0.18cm of \i EE] (\i EEN) ;
		\coordinate[below =0.18cm of \i EE] (\i EES) ;
		\coordinate[right =0.18cm of \i EE] (\i EEE) ;
	\node (\i ES)[below = 0.5cm of \i E]{} ;
		\coordinate[below =0.18cm of \i ES] (\i ESS) ;
		\coordinate[left =0.18cm of \i ES] (\i ESE) ;
		\coordinate[right =0.18cm of \i ES] (\i ESW) ;
	\node (\i EN)[above = 0.5cm of \i E]{} ;
		\coordinate[above =0.18cm of \i EN] (\i ENN) ;
		\coordinate[left =0.18cm of \i EN] (\i ENE) ;
		\coordinate[right =0.18cm of \i EN] (\i ENW) ;
\node (\i W)[left = 1cm of \i ]{} ;
	\node (\i WS)[below = 0.5cm of \i W]{} ;
		\coordinate[below =0.18cm of \i WS] (\i WSS) ;
		\coordinate[left =0.18cm of \i WS] (\i WSE) ;
		\coordinate[right =0.18cm of \i WS] (\i WSW) ;
	\node (\i WN)[above = 0.5cm of \i W]{} ;
		\coordinate[above =0.18cm of \i WN] (\i WNN) ;
		\coordinate[left =0.18cm of \i WN] (\i WNE) ;
		\coordinate[right =0.18cm of \i WN] (\i WNW) ;
	\node (\i WW)[left = 0.5cm of \i W]{} ;
		\coordinate[above =0.18cm of \i WW] (\i WWN) ;
		\coordinate[below =0.18cm of \i WW] (\i WWS) ;
		\coordinate[left =0.18cm of \i WW] (\i WWW) ;
\graph[edge={thick,color=red}]{
	(0)--(\i )--(\i\i)--(\i\i\i)--[densely dotted](\i\i\i\i),
	(\i ENN)--[densely dotted](\i EN)--(\i E)--(\i ES)--[densely dotted](\i ESS),
	(\i WNN)--[densely dotted](\i WN)--(\i W)--(\i WS)--[densely dotted](\i WSS),
	(\i EES)--[densely dotted](\i EE)--[densely dotted](\i EEN),
	(\i WWS)--[densely dotted](\i WW)--[densely dotted](\i WWN),
	(\i\i ES)--[densely dotted](\i\i E)--[densely dotted](\i\i EN),
	(\i\i WS)--[densely dotted](\i\i W)--[densely dotted](\i\i WN)};
\graph[edge={thick,color=blue}]{
	(\i WWW)--[densely dotted](\i WW)--(\i W)--(\i)--(\i E)--(\i EE)--[densely dotted](\i EEE),
	(\i\i WW)--[densely dotted](\i\i W)--(\i\i)--(\i\i E)--[densely dotted](\i\i EE),
	(\i ENW)--[densely dotted](\i EN)--[densely dotted](\i ENE),
	(\i ESW)--[densely dotted](\i ES)--[densely dotted](\i ESE),
	(\i WNW)--[densely dotted](\i WN)--[densely dotted](\i WNE),
	(\i WSW)--[densely dotted](\i WS)--[densely dotted](\i WSE),
	(\i\i\i E)--[densely dotted](\i\i\i)--[densely dotted](\i\i\i W)};
}
}
\end{tikzpicture}}
%\caption{Lorem ipsum}
\end{subfigure}
\caption{Fragments of the Cayley graphs of $\Z^2$ ($1$ end) on the left and of $F_2$ (infinitely many ends) on the right; with standard generating sets.}
\label{Figure:CayleyOfZ2}
\end{figure}
\begin{figure}[htbp]\centering
\begin{tikzpicture}[node distance=1.14cm,every node/.style=blacknode]
\node (00){} ;
\coordinate[left =0.5cm of 00] (0a) ;
\node (01)[right of=00]{} ;
\node (02)[right of=01]{} ;
\node (03)[right of=02]{} ;
\node (04)[right of=03]{} ;
\coordinate[right =0.5cm of 04] (0b) ;
\node (10)[above of=00]{} ;
\coordinate[left =0.5cm of 10] (1a) ;
\node (11)[above of=01]{} ;
\node (12)[above of=02]{} ;
\node (13)[above of=03]{} ;
\node (14)[above of=04]{} ;
\coordinate[right =0.5cm of 14] (1b) ;
\graph[edge={thick,color=blue}]{(0a)--[dotted](00)--(01)--(02)--(03)--(04)--[dotted](0b)} ;
\graph[edge={thick,color=blue}]{(1a)--[dotted](10)--(11)--(12)--(13)--(14)--[dotted](1b)} ;
\graph[edge={thick,color=red}]{(02)--(12)} ;
\foreach \i in {0,1,3,4}{
\draw [thick,color=red] (0\i) to [myloop below] (0\i) ;
\draw [thick,color=red] (1\i) to [myloop above] (1\i) ;
}
%\begin{scope}[yshift=-1cm]
%\node (0){} ;
%%\coordinate[left =0.5cm of 0] (a) ;
%\coordinate[left =1cm of 0] (-1) ;
%\coordinate[left =2cm of 0] (-2) ;
%\coordinate[left =3cm of 0] (-3) ;
%\node (1)[right of=0]{} ;
%\node (2)[right of=1]{} ;
%\node (3)[right of=2]{} ;
%%\node (4)[right of=3]{} ;
%%\coordinate[right =0.5cm of 4] (b) ;
%\coordinate[right =1cm of 3] (4) ;
%\coordinate[right =2cm of 3] (5) ;
%\coordinate[right =3cm of 3] (6) ;
%\graph[edge={thick}]{(-2)--[bend left,red](0)--[bend left,red](2)--[bend left,red](4), (-1)--[bend left,red](1)--[bend left,red](3)--[bend left,red](5) } ;
%\graph[edge={thick}]{(-3)--[bend right,blue](0)--[bend right,blue](3)--[bend right,blue](6), (-2)--[bend right,blue](1)--[bend right,blue](4),(-1)--[bend right,blue](2)--[bend right,blue](5)} ;
%\end{scope}
\end{tikzpicture}
\caption{A fragment of a Schreier graph ($4$ ends) of the free group $F_2=\langle \textcolor{red}{x^{\pm1}},\textcolor{blue}{y^{\pm1}}\rangle$.}
\label{Figure:SchreierOfF2}
\end{figure}
%
%

Let $\Gamma$ be a graph and $K$ a finite subset of vertices. The graph $\Gamma\setminus K$ is the subgraph of $\Gamma$ obtained by deleting all vertices in $K$ and all edges adjacent to them. This graph is not necessarily connected.
\begin{defn}\label{Def:Ends}
Let $\Gamma$ be a graph. The \emph{number of ends} of $\Gamma$ is the supremum, taken over all finite $K$, of the number of infinite connected components of $\Gamma\setminus K$.
\end{defn}
There exists other characterizations of the number of ends in graphs, %notably in terms of equivalence classes of rays,
see \cite{MR1967888} and the references therein, but Definition \ref{Def:Ends} is the one that best suits our purpose.
%The interested reader may refer to \cite{MR1967888} and the references therein for an extended treatment of the
A locally finite graph (i.e. such that every vertex has finite degree) is finite if and only if it has $0$ end.
%The graphs of the Figures~\ref{Figure:CayleyOfZ} and \ref{Figure:CayleyOfZ2} have respectively 2 ends and 1 end.

An important fact about the number of ends of a graph, is that it is an invariant of quasi-isometry, see \cite{MR1213151}. In particular, if $G$ is a finitely generated group it is possible to speak about the number of ends of the Schreier graph $\Sch(G,H;S)$ without specifying a particular finite generating set $S$.
By a celebrated result of Hopf \cite{MR10267}, the number of ends of a Cayley graph of a finitely generated group can only be $0$, $1$, $2$ or infinite (in which case it is uncountable), see Figures~\ref{Figure:CayleyOfZ} and \ref{Figure:CayleyOfZ2} for some examples.
On the other hand, Schreier graphs may have any number of ends in $\mathbf N\cup\{\infty\}$, see Figure~\ref{Figure:SchreierOfF2} for an example of a graph with $4$ ends.
%While the number of ends of Cayley graphs of finitely generated groups can only be $0$, $1$, $2$ or infinite (in which case it is uncountable) \cite[Theorem 11.27]{Stallings1971}, Schreier graphs may have any number of ends in $\mathbf N\cup\{\infty\}$, see Figure~\ref{Figure:SchreierOfF2} for an example of a graph with another number of ends.
In fact, every regular graph of even degree is isomorphic to a Schreier graph, \cite{MR0450121,MR1358635}.

We are now finally able to introduce property FW. Instead of giving the original definition in terms of actions on wall spaces, we will use an equivalent one for finitely generated groups due to Cornulier \cite{Cornulier2013}.
%
%
\begin{defn}
A finitely generated group $G$ has \emph{property FW} if all its Schreier graphs have at most one end.
\end{defn}
%
%
It directly follows from the definition that all finite groups have property FW, but that $\Z$ does not have it.
In fact, if $G$ is a finitely generated group with an homomorphism onto $\Z$, then it does not have FW. Indeed, in this case $G\cong H\rtimes \Z$  for some $H$ and the Schreier graph $\Sch(G,H;S)$ is isomorphic to a Cayley graph of $\Z\cong G/H$ and hence has $2$ ends.

Property FW admits many distinct characterizations that allow to define it for groups that are non-necessarily finitely generated and even for topological groups. We refer the reader to \cite{Cornulier2013} for a survey of these characterizations.
%
%
%
%
%
%
%
%
%
%
%%%%%%%%%%%%%%%%%%%%%%%%%%%%%%%%%%%%%%%%%%%%%%%%%%%%%%%%%%%%%%%%%%%%%%%%%%%
%%%%%%%%%%%%%%%%%%%%%%%%%%%%%%%%%%%%%%%%%%%%%%%%%%%%%%%%%%%%%%%%%%%%%%%%%%%
%%%%%%%%%%%%%%%%%%%%    Subsection : Wreath products    %%%%%%%%%%%%%%%%%%%
%%%%%%%%%%%%%%%%%%%%%%%%%%%%%%%%%%%%%%%%%%%%%%%%%%%%%%%%%%%%%%%%%%%%%%%%%%%
%%%%%%%%%%%%%%%%%%%%%%%%%%%%%%%%%%%%%%%%%%%%%%%%%%%%%%%%%%%%%%%%%%%%%%%%%%%
\subsection{Wreath products}
%
%
%
%
%
Let $X$ be a set and $G$ a group. We view
$\bigoplus_XG$ as the set of functions from $X$ to $G$ with finite support:
\[
	\bigoplus_XG=\setst{\varphi\colon X\to G}{\varphi(x)=1 \textnormal{ for all but finitely many }x}.
\]
This is naturally a group, where multiplication is taken componentwise.

If $H$ is a group acting on $X$, then it naturally acts on $\bigoplus_XG$
by $(h.\varphi)(x)=\varphi(h^{-1}.x)$.
This leads to the following standard definition.
\begin{defn}\label{Def:WreathProd}
Let $G$ and $H$ be groups and $X$ be a set on which $H$ acts.
The \emph{(retricted) wreath product} $G\wr_XH$ is the group $(\bigoplus_XG)\rtimes H$.
\end{defn}
A prominent  source of examples of wreath products is the ones of the form $G\wr_HH$, where $H$ acts on itself by left multiplication.
In particular, the group $(\Z/2\Z)\wr_\Z\Z$ has become well-known under the name of the \emph{lamplighter group}.
Other (trivial) examples of wreath products are direct products $G\times H$ which corresponds to wreath products over a singleton $G\wr_{\{*\}}H$.

Let $S$ be a generating set of $G$ and $T$ a generating set of $H$.
Let $\{x_i\}_{i\in I}$ be a choice of a representative in each $H$-orbit.
Finally, let $\delta_x^s$ be the element of $\bigoplus_XG$ defined by $\delta_x^s(x)=s$ and $\delta_x^s(y)=1_G$ if $y\neq x$ and let $\mathbf 1$ be the constant function with value $1_G$.
It is then standard that
\[
	\setst{(\delta_{x_i}^s,1_H)}{s \in S,i\in I} \cup \setst{(\mathbf 1,t)}{t \in T}
\]
is a generating set for $G\wr_XH$.

On the other hand, if $\setst{(\varphi_i,h_i)}{i\in I}$ is a generating set of $G\wr_XH$, then $\setst{h_i}{i\in I}$ is a generating set of $H$ while $\setst{\varphi_i(x)}{i\in I,x\in X}$ is a generating set of $G$.
Observe that since the $\varphi_i$ take only finitely many values, if $I$ is finite, so is $\setst{\varphi_i(x)}{i\in I,x\in X}$.
We hence recover the following characterization of the finite generation of $G\wr_XH$.

%On the other hand, if $G\wr_XH$ is finitely generated, then its quotient $H$ is also finitely generated.
%It follows from the definition of the wreath product 
%that
%\[
%	G\wr_XH\cong(\bigoplus_{Y\in X/H}L_Y)\rtimes H\qquad\textnormal{with}\qquad L_Y\cong\bigoplus_{y\in Y}G_y,
%\]
%where $X/H$ is the set of $H$-orbits.
%The important fact for us is that $H$ fixes the decomposition into $L_Y$ factors: for all $Y$ we have $H.L_Y=L_Y$.
%We hence obtain that if $G\wr_XH$ is finitely generated, then $H$ acts on $X$ with finitely many orbit.
%Finally\todo{Prendre un système de gen de $G\wr_XH$ et en extraire un système de gen de $G$...}
%
%
\begin{lem}
The group $G\wr_XH$ is finitely generated if and only if both $G$ and $H$ are finitely generated and $H$ acts on $X$ with finitely many orbits.
\end{lem}
\begin{proof}
%The only implication that does not directly follows of the discussion above is that if $G\wr_XH$ is finitely generated, then $H$ acts on $X$ with finitely many orbits.
If $G\wr_XH$ is finitely generated, so is its abelianization $(G\wr_XH)^{\ab}\cong (\bigoplus_{X/H}G^{\ab})\times H^{\ab}$, which implies that the orbit set $X/H$ is finite. The other implications directly follow from the above discussion on generating sets.
\end{proof}
%
%
Using the above lemma, we could reformulate Theorem~\ref{Thm:Main} in the following way:
%
%
\begin{prop}
Let $G$, $H$ be two groups with $G$ non-trivial and $X$ a set on which $H$ acts. Suppose that all three of $G$, $H$ and $G\wr_XH$ are finitely generated. Then the wreath product $G\wr_XH$ has property FW if and only if $G$ and $H$ have property FW and $X$ is finite.
\end{prop}
