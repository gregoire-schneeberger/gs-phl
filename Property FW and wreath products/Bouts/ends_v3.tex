\documentclass[a4paper]{article}
\usepackage[utf8]{inputenc}
\usepackage[T1]{fontenc}      
\usepackage[english]{babel} %utilisation du package français
%
%
%
\usepackage{textcomp}% améliore certains symboles de bases
\usepackage{lmodern}% remplace la police ComputerModern par LatinModern (+ mieux bien)
%
%
%
\usepackage{mathtools}% compléments à amsmath
\usepackage{amssymb,amsfonts}% symboles mathématiques supplémentaires
\usepackage{amsthm}% environnement ams-theorem
\usepackage[colorlinks,breaklinks,bookmarks,plainpages=false,unicode=true]{hyperref}%
\usepackage{todonotes} %ajouter des commentaires
%\usepackage[disable]{todonotes}%cache les commentaires
\usepackage{datetime} %ajoute la date et l'heure
%
%
\newcounter{mycomment}
\newcommand{\mycomment}[2][]{\refstepcounter{mycomment}{\todo[color={green!33},size=\small]{\textbf{[\uppercase{#1}\themycomment]:}~#2}}}
\newcommand{\PH}[1]{\todo[color={blue!33},size=small]{\textbf{PH :} #1}}
\newcommand{\PHInline}[1]{\todo[color={blue!33},size=small,inline]{\textbf{PH :} #1}}
\newcommand{\GS}[1]{\mycomment[GS]{#1}}
%
\usepackage{subcaption}
\usepackage{tikz}
\usetikzlibrary{external,graphs}
\tikzexternalize[prefix=Images/]% On externalise. Les images sont stockées dans le dossier Images
\makeatletter %pour ne pas prendre en compte les todo dans l'externalisation
\renewcommand{\todo}[2][]{\tikzexternaldisable\@todo[#1]{#2}\tikzexternalenable}
\makeatother
\makeatletter %pour ne pas prendre en compte les missingfigures dans l'externalisation
\renewcommand{\missingfigure}[2][]{\tikzexternaldisable\@missingfigure[#1]{#2}\tikzexternalenable}
\makeatother
\tikzset{myloop above/.style={loop, out=130, in = 50,min distance =8mm}}
\tikzset{myloop below/.style={loop, out=-130, in = -50,min distance =8mm}}
\tikzset{blacknode/.style={shape= circle, fill = black, inner sep = 0pt, outer sep = 0pt, minimum size = 3pt,draw}}
%
\newtheorem{lem}{Lemma}[section]
\newtheorem{conjecture}[lem]{Conjecture}
\newtheorem{cor}[lem]{Corollary}
\newtheorem{prop}[lem]{Proposition}
\newtheorem{thm}[lem]{Theorem}
\theoremstyle{definition}
\newtheorem{defn}[lem]{Definition}
\theoremstyle{remark}% texte en roman
\newtheorem{exmp}[lem]{Example}
%
%
%
\DeclareMathOperator\Cayley{Cayl}
\DeclareMathOperator\Sch{Sch}
\DeclareMathOperator\stab{Stab}
\DeclareMathOperator\ab{ab}
%
\DeclarePairedDelimiter\abs{\lvert}{\rvert}
\newcommand*{\field}[1]{\mathbf{#1}}
\newcommand*{\Z}{\field{Z}}
\newcommand{\setst}[2]{\{#1\ |\ #2\}}
%
%
\title{Property FW and wreath products of groups: a~simple approach using Schreier graphs}
\author{Paul-Henry Leemann\thanks{Supported by grant 200021\textunderscore188578 of the Swiss National Fund for Scientific Research.}, Grégoire Schneeberger}
\date{\today \quad \currenttime}
%
%
%
%
%
%
%
%
%
%
\begin{document}
\maketitle
%
%
%
%
%
%
%
%
%
%
\begin{abstract}
The group property FW stands in-between the celebrated Kazdhan's property (T) and Serre's property FA. Among many characterizations, it might be defined, for finitely generated groups, as having all Schreier graphs one-ended.
%Yves Cornulier proved that a finitely generated wreath product $G\wr_XH$ has property FW if and only if both $G$ and $H$ have property FW and $X$ is finite. In this paper, we will present a more explicit combinatorial elementary proof using Schreier graphs. \GS{Ajouté que c'est déjà dans Cornulier}

It follows from the work of Y. Cornulier that a finitely generated wreath product $G\wr_XH$ has property FW if and only if both $G$ and $H$ have property FW and $X$ is finite.
%In this paper, we will present a more explicit combinatorial elementary proof using Schreier graphs.
The aim of this paper is to give an elementary, direct and explicit proof of this fact using Schreier graphs.\PH{Légèrement reformulé.}
\end{abstract}
%
%
%
%
%
%
%
%
%
%
%
%
%
%
%
%!TEX root = ends_v3.tex
%
%
%
%
%
%
%
%
%
%
%%%%%%%%%%%%%%%%%%%%%%%%%%%%%%%%%%%%%%%%%%%%%%%%%%%%%%%%%%%%%%%%%%%%%%%%%%%
%%%%%%%%%%%%%%%%%%%%%%%%%%%%%%%%%%%%%%%%%%%%%%%%%%%%%%%%%%%%%%%%%%%%%%%%%%%
%%%%%%%%%%%%%%%%%%%%%%%    Section : Introduction    %%%%%%%%%%%%%%%%%%%%%%
%%%%%%%%%%%%%%%%%%%%%%%%%%%%%%%%%%%%%%%%%%%%%%%%%%%%%%%%%%%%%%%%%%%%%%%%%%%
%%%%%%%%%%%%%%%%%%%%%%%%%%%%%%%%%%%%%%%%%%%%%%%%%%%%%%%%%%%%%%%%%%%%%%%%%%%
\section{Introduction}
%
%
%
%
%
Property FW is a group property that is (for discrete groups) a weakening of the celebrated Kazdhan's property (T). It was introduced by Barnhill and Chatterji in \cite{Barnhill2008}. It is a fixed point property for actions on wall spaces or, equivalently, on CAT(0) cube complexes. Therefore it stands between property FH (fixed points on Hilbert spaces, equivalent to property (T) for discrete groups) and property FA (fixed points on \emph{arbres}\footnote{\emph{Arbres} is the french word for trees.}). It is known that all these properties are different, see  \cite{Cornulier2013} for examples of groups which distinguish them.

When working with group properties, it is natural to ask if they are stable under ``natural'' group operations. One such operation, of great use in geometric group theory, is the wreath product, which generalizes the direct product of two groups, see Definition~\ref{Def:WreathProd}.
%\todo{Est ce que c'est écrit quelques part les preuves pour cela ? Si oui citer si non les écrire pour être complet}
%A slightly less common operation, but still of great use in geometric group theory, is the wreath product, see Definition~\ref{Def:WreathProd}.
%\todo[inline]{Comme dit dans le mail, la stabilité par sous-groupe et produit semi-direct est fausse (cf \cite{Cornulier2013} et les références dedans pour des examples). Dans la suite, on (re)montre que c'est stable par quotient et produit direct (\cite{Cornulier2013} montre que pour $1\to N\to G\to H\to 1$ on a $G$ a FW ssi $H$ a FW et $(G,N)$ a relatif FW; ça implique la stabilité par produit et quotient.)
%
%Le mieux est peut-être de réécrire la phrase comme ceci:
%}

In the context of properties defined by fixed points of actions, the first result concerning wreath products is due to Cherix, Martin and Valette and later refined by Neuhauser and concerns property (T).
%
%
\begin{thm}[\cite{Cherix2004,Neuhauser2005a}] \label{T:Wreath_prop_T}
Let $G,H$ be two discrete groups with $G$ non-trivial and $X$ a set on which $H$ acts. The wreath product $G \wr_X H$ has property (T) if and only if $G$ and $H$ have property (T) and $X$ is finite.
\end{thm}
%
%
As consequence of more general results in \cite{Cornulier2013}, Cornullier proved an analogous of Theorem~\ref{T:Wreath_prop_T} for property FW: \GS{Màj avec Cornullier}
%
%
\begin{thm}\label{Thm:Main}
Let $G,H$ be two finitely generated groups with $G$ non-trivial and $X$ a set on which $H$ acts with finitely many orbits. The wreath product $G \wr_X H$ has property FW if and only if $G$ and $H$ have property FW and $X$ is finite.
\end{thm}
%
%
 The aim of this note is to give an explicit combinatorial elementary proof using a characterization of property FW with the number of ends of Schreier graphs, see Subsection \ref{Subsection:FW} for the relevant definitions. \GS{Idem}

%Corulier and Kar studied the question of stability of wreath product for property FA in \cite{Cornulier2011}. They proved a similar result if the action of the group $H$ has finitely many orbits and no fixed point. The condition that $G$ has property FA is replaced by $G$ has finite abelianization and $G$ can not be expressed as a countable union of properly increasing sequence of subgroups. %\todo{Peut être mettre une phrase qui explique en quoi c'est différent de FA}

%A somewhat similar result for property FA was obtained by Cornulier and Kar in \cite{Cornulier2011}.
\GS{Enlevé la discussion pour FA, car on veut peut être centrer sur la preuve, mais on peut remettre}
At this point, the curious reader might have two questions. First, is it possible to extend Theorem \ref{Thm:Main} beyond the realms of finitely generated groups and of actions with finitely many orbits? And secondly, is there a link between Theorems \ref{T:Wreath_prop_T} and \ref{Thm:Main}?
In both cases, the answer is yes.
This is the subject of the forthcoming and more technical paper \cite{LS2021}, which gives an unified proof of Theorems \ref{T:Wreath_prop_T} and \ref{Thm:Main} as well as of similar results for the Bergman's property and more.
%
%
%
%
\paragraph{Organization of the paper}
The next section contains all the definitions as well as some examples, while Section \ref{Section:Proof} is devoted to the proof of Theorem~\ref{Thm:Main} and some related results.
%
%
%
%\todo[inline]{Mettre ici les remerciements. Tu dois possiblement remercier la NSF. J'ai mis Alain en remerciement, je ne sais pas ce que tu en penses et si tu veux être plus spécifique.}
\paragraph{Acknowledgment}
The authors are thankful to A. Genevois, T. Nagnibeda and A. Valette for helpful comments on a previous version of this note, and to the Swiss National Fund for Scientific Research for its support.
% The second author thanks T. Nagnibeda for her comments and the NSF for its support.
%
%
%
%\todo[inline]{J'ai mis en commentaire la partie sur les actions essentielles ne sachant pas comment l'introduire. Je pense que si on veut la garder, le mieux est de la mettre un peu en apparté soit à la fin de l'intro soit à la fin de l'article. Cela permet de garder quelque chose de simple et d'auto-contenu pour le reste.}

%An action of a group on a CAT(0) cube complex is \emph{essential} if all the orbits of vertices are unbounded and the action is transitive on the set of hyperplanes.
%\begin{cor}
%Let $G,H$ be two discrete groups and $X$ a set on which $H$ acts transitively. If there exists an essential action of $G$ or $H$ on a CAT(0) cube complex  or if $X$ is infinite, then there exists an essential action of $G \wr_X H$ on a CAT(0) cube complex. 
%\end{cor}
\input{Definitions_v3.tex}
\input{Proof_v3.tex}
%
%
%
%
%
%
%
%
%
%
\begin{thebibliography}{10}

\bibitem{Barnhill2008}
Angela Barnhill and Indira Chatterji.
\newblock Property ({T}) versus property {FW}.
\newblock {\em Enseign. Math. (2)}, 54(1-2):16--18, 2008.

\bibitem{MR1213151}
Stephen~G Brick.
\newblock {Quasi-isometries and ends of groups}.
\newblock {\em J. Pure Appl. Algebr.}, 86(1):23--33, 1993.

\bibitem{Cherix2004}
Pierre-Alain Cherix, Florian Martin, and Alain Valette.
\newblock Spaces with measured walls, the {H}aagerup property and property
  ({T}).
\newblock {\em Ergodic Theory Dynam. Systems}, 24(6):1895--1908, 2004.

\bibitem{Cornulier2013}
Yves {Cornulier}.
\newblock {Group actions with commensurated subsets, wallings and cubings}.
\newblock {\em arXiv e-prints}, page arXiv:1302.5982, February 2013.

\bibitem{Cornulier2011}
Yves Cornulier and Aditi Kar.
\newblock On property ({FA}) for wreath products.
\newblock {\em J. Group Theory}, 14(1):165--174, 2011.

\bibitem{DelaHarpe2000}
Pierre de~la Harpe.
\newblock {\em Topics in geometric group theory}.
\newblock Chicago Lectures in Mathematics. University of Chicago Press,
  Chicago, IL, 2000.

\bibitem{MR1967888}
Reinhard Diestel and Daniela K\"{u}hn.
\newblock Graph-theoretical versus topological ends of graphs.
\newblock {\em J. Combin. Theory Ser. B}, 87(1):197--206, 2003.

\bibitem{MR0450121}
Jonathan~L. Gross.
\newblock Every connected regular graph of even degree is a {S}chreier coset
  graph.
\newblock {\em J. Combinatorial Theory Ser. B}, 22(3):227--232, 1977.

\bibitem{MR10267}
Heinz Hopf.
\newblock Enden offener {R}\"{a}ume und unendliche diskontinuierliche
  {G}ruppen.
\newblock {\em Comment. Math. Helv.}, 16:81--100, 1944.

\bibitem{LS2021}
Paul-Henry Leemann and Gr{\'e}goire Schneeberger.
\newblock Wreath products of groups acting with bounded orbits.
\newblock {\em arXiv e-prints}, page arXiv:2102.08001, February 2021.

\bibitem{MR1358635}
Alexander Lubotzky.
\newblock Cayley graphs: eigenvalues, expanders and random walks.
\newblock In {\em Surveys in combinatorics, 1995 ({S}tirling)}, volume 218 of
  {\em London Math. Soc. Lecture Note Ser.}, pages 155--189. Cambridge Univ.
  Press, Cambridge, 1995.

\bibitem{Neuhauser2005a}
Markus Neuhauser.
\newblock Relative property ({T}) and related properties of wreath products.
\newblock {\em Math. Z.}, 251(1):167--177, 2005.

\bibitem{MR1347406}
Michah Sageev.
\newblock Ends of group pairs and non-positively curved cube complexes.
\newblock {\em Proc. London Math. Soc. (3)}, 71(3):585--617, 1995.

\end{thebibliography}

%
%
%
%
%
\enddocument