% !TEX spellcheck = English (aspell)

\documentclass[a4paper]{article}
\usepackage[utf8]{inputenc}
\usepackage[T1]{fontenc}      
\usepackage[english]{babel} %utilisation du package français
%
\usepackage{textcomp}% améliore certains symboles de bases
\usepackage{lmodern}% remplace la police ComputerModern par LatinModern (+ mieux bien)
%
\usepackage[a4paper]{geometry} % Pour avoir les marges pour du papier A4
%
\usepackage[intlimits, leqno]{amsmath}% ams-mathh ! avec numérotation des équations à gauche et intégrales avec limites dessous
\usepackage{mathtools}% compléments à amsmath
\usepackage{amssymb,amsfonts}% symboles mathématiques supplémentaires
\usepackage{bm}% symbole math gras
\usepackage{amsthm}% environnement ams-theorem
\usepackage{array}% ajoute des options aux environnements tabular et arrray, permet de centrer
\usepackage{mathrsfs}% permet d'utiliser matcal
\usepackage[colorlinks,breaklinks,bookmarks,plainpages=false,unicode=true]{hyperref}%
\usepackage{todonotes} %ajouter des commentaires
%\usepackage[disable]{todonotes}%cache les commentaires
\usepackage{datetime} %ajoute la date et l'heure
\usepackage{here} % permet d'utilser H, h pour place les figures
%
%
%
\newtheorem{lem}{Lemma}[section]
%\newtheorem{AFaire}[lemma]{A faire}% peut enlever à la fin
%\newtheorem{Next}[lemma]{Possible continuation}% peut enlever à la fin
\newtheorem{conjecture}[lem]{Conjecture}
\newtheorem{cor}[lem]{Corollary}
\newtheorem{prop}[lem]{Proposition}
\newtheorem{question}[lem]{Question}
\newtheorem{thm}[lem]{Theorem}
\theoremstyle{remark}% texte en roman
\newtheorem{claim}[lem]{Affirmation}
\newtheorem{defn}[lem]{Definition}
\newtheorem{exmp}[lem]{Example}
\newtheorem{fact}[lem]{Fact}
\newtheorem{notation}[lem]{Notation}
\newtheorem{rem}[lem]{remark}
\newtheorem{scholion}[lem]{Scholion}
\newtheorem{exo}{Exercise}
%
%
%
\DeclareMathOperator\Aut{Aut} 
\DeclareMathOperator\cl{Cl}
\DeclareMathOperator\fix{Fix}
\DeclareMathOperator\GL{GL}		%groupe linéaire
\DeclareMathOperator\im{im}
\DeclareMathOperator\id{Id}
\DeclareMathOperator\orbite{O}
\DeclareMathOperator\ppcm{ppcm}
\DeclareMathOperator\pgcd{pgcd}
\DeclareMathOperator\Sch{Sch}
\DeclareMathOperator\sign{sign}
\DeclareMathOperator\SL{SL}		
\DeclareMathOperator\stab{Stab}
\DeclareMathOperator\supp{supp}
%
\DeclarePairedDelimiter\abs{\lvert}{\rvert}
\DeclarePairedDelimiter\gen{\langle}{\rangle}
%
\newcommand*{\actsGroup}{\curvearrowright}% sementic command for G acts X
\newcommand*{\iso}{\cong}% isomorphisme,commande sémantique (\simeq ou \cong)
\renewcommand*{\H}{\mathcal{H}}
\renewcommand*{\phi}{\varphi}
%
\newcommand*{\field}[1]{\mathbf{#1}}
\newcommand*{\N}{\field{N}}
\newcommand*{\Z}{\field{Z}}
\newcommand*{\Q}{\field{Q}}
\newcommand*{\R}{\field{R}}
\renewcommand*{\C}{\field{C}}
%
%
%
\title{The Property FW for the wreath products}
\author{Paul-Henry Leemann, Grégoire Schneeberger}
\date{\today \quad \currenttime}
%
%
%
\begin{document}
\maketitle

\section{Introduction}
%
The property FW was introduced by de Cornullier. It is a fixed point property for the action on wall spaces (for a detailed treatment of this property see \cite{Cornulier2013}). For discrete groups, this property is implied by the Kazhdan property (T). The behavior of the Kazhdan Property (T) with the wreath product is well known: 
\begin{thm}[\cite{Cherix2004,Neuhauser2005a}]
Let $G,H$ be two discrete groups and $X$ a set on which $H$ acts. The wreath product $G \wr_X H$ has the property (T) if and only if $G$ and $H$ have the property (T) and if $X$ is finite.
\end{thm}
%
%
The same kind of result is true for the property FW
\begin{thm}
Let $G,H$ be two discrete groups and $X$ a set on which $H$ acts transitively. The wreath product $G \wr_X H$ does not have the property FW if at least one of the following conditions is satisfied: 
\begin{enumerate}
\item The group $G$ does not have the property FW.
\item The group $H$ does not have the property FW.
\item The set $X$ is infinite.
\end{enumerate}
\end{thm}
%
%
An action of a group on a CAT(0) cube complex is \emph{essential} if all the orbits of vertices are unbounded and the action is transitive on the set of hyperplanes.
\begin{cor}
Let $G,H$ be two discrete groups and $X$ a set on which $H$ acts transitively. If there exists an essential action of $G$ or $H$ on a CAT(0) cube complex  or if $X$ is infinite, then there exists an essential action of $G \wr_X H$ on a CAT(0) cube complex. 
\end{cor}
%
%

%
%
%------------------------------------Definitions----------------------------------
\section{Definitions}
\subsection{The Property FW}
%
For the definition of the property FW, we will follow the survey of Y. de Cornullier \cite{Cornulier2013}.
%
\begin{defn}
Let $G$ be a discrete group and $X$ a discrete set on which $G$ acts. A subset $M \subset X$ is \emph{commensurated} by the $G$-action if 
\begin{equation*}
\abs{gM \Delta M} < \infty 
\end{equation*}
for all $g$ in $G$.
\end{defn}
An invariant $G$-subset is automatically  commensurated. Moreover, for a subset $M$ such that there exists an invariant $G$-subset $N$ with $\abs{M \Delta N}< \infty$ then $M$ is commensurated. 
%Indeed: 
%\todo{écrire une preuve}
Such a set is called \emph{transfixed}.
%
%
\begin{defn}
A group $G$ has the property FW if all commensurable $G$-set are transfixed.
\end{defn}
There are lot of equivalent characterizations of this property. We will give us without all the details and the precise definitions.
%
%
\begin{prop}\label{P:charact_FW}
The following are equivalent:
\begin{enumerate}
\item\label{C:FW_def} $G$ has the property FW;
\item every cardinal definite function on $G$ is bounded;
\item\label{C:FW_Cat(0)} every cellular action on a CAT(0) cube complex has bounded orbits for the $\ell^1$-metric (the complexes can be infinite dimensional);
\item every cellular action on a CAT(0) cube complex has a fixed point;
\item \label{C_FW_MedianGRaph}every action on a connected median graph has bounded orbits;
\item every action on a nonempty connected median graph has a fixes point;
\item \label{C:FW_Schrei} (if $G$ is finitely generated) every Schreier graph of $G$ has at most 1 end;
\item \label{C:FW_Wallings}For every set $Y$ endowed with a walling structure and compatible action on $Y$ and on the index of the walling, the action on $Y$ has bounded orbits for the wall distance;
\item \label{C:FW_Hilbert} every isometric action on an "integral Hilbert space" $\ell^2(X,\Z)$ (X any discrete set), or equivalently on $\ell^2(X,\R)$ preserving the integral points, has bounded orbits;
\item for every $G$-set $X$ we have $H^1(G,\Z X)=0$.
\end{enumerate}
\end{prop}
Note that the name FW comes from the property of "fixed point" for the actions on the walling spaces. We will see in the following that a semi-splittable group does not have the property FW (see corollary \ref{C:SS_FW}).
%

The property FW has links with other well known properties. For example, the property FH implies the characterisation \ref{C:FW_Hilbert}. For discrete groups (and even for countable groups) the property FH is equivalent to the Kazhdan's property (T) by Delorme-Guichardet's Theorem. As trees are CAT(0) cube complexes, the property FW implies Serre's property FA. %
%
%
\subsection{Graphe de Schreier}
%
%
%
\subsection{Bouts}
%
%
\section{Proof of the Theorem}


\bibliography{biblio.bib}
\bibliographystyle{abbrv}


\enddocument