\documentclass[a4paper]{article}
\usepackage[utf8]{inputenc}
\usepackage[T1]{fontenc}
\usepackage[english]{babel} %utilisation du package français
%
%
%
\usepackage{textcomp}% améliore certains symboles de bases
\usepackage{lmodern}% remplace la police ComputerModern par LatinModern (+ mieux bien)
%
%
%
\usepackage{mathtools}% compléments à amsmath
\usepackage{amssymb,amsfonts}% symboles mathématiques supplémentaires
\usepackage{amsthm}% environnement ams-theorem
\usepackage[colorlinks,breaklinks,bookmarks,plainpages=false,unicode=true]{hyperref}%
\usepackage{todonotes} %ajouter des commentaires
\newcounter{mycomment}
\newcommand{\mycomment}[2][]{\refstepcounter{mycomment}{\todo[color={green!33},size=\small]{\textbf{Commentaire [\uppercase{#1}\themycomment]:}~#2}}}
\newcommand{\PH}[1]{\todo[color={blue!33},size=small]{#1}}
\newcommand{\GS}[1]{\mycomment[GS]{#1}}
%
%
%
\usepackage{datetime} %ajoute la date et l'heure
%
%
%
\newtheorem{lem}{Lemma}[section]
%\newtheorem{conjecture}[lem]{Conjecture}
\newtheorem{cor}[lem]{Corollary}
\newtheorem{prop}[lem]{Proposition}
\newtheorem{thm}[lem]{Theorem}
%\newtheorem{question}[lem]{Question}
\theoremstyle{definition}
\newtheorem{defn}[lem]{Definition}
\newtheorem{rem}[lem]{Remark}
\newtheorem{exmp}[lem]{Example}
%
%
%
%\DeclareMathOperator\Cayley{Cayl}
%\DeclareMathOperator\Sch{Sch}
%\DeclareMathOperator\stab{Stab}
\DeclareMathOperator\Ind{Ind}
\DeclareMathOperator\SL{SL}
\DeclareMathOperator\Sym{Sym}
\DeclareMathOperator\diag{diag}
%\DeclareMathOperator\diameter{diam}
%\DeclareMathOperator\Hom{Hom}
%\DeclareMathOperator\Isom{Isom}
\DeclareMathOperator\Aut{Aut}
%\DeclareMathOperator\id{id}
\DeclareMathOperator\ab{ab}
\DeclareMathOperator\diam{diam}
%
%\DeclarePairedDelimiter\abs{\lvert}{\rvert}
%\DeclarePairedDelimiter\gen{\langle}{\rangle}
%
\newcommand*{\orbite}{\mathcal O}
\newcommand*{\field}[1]{\mathbf{#1}}
\newcommand*{\Z}{\field{Z}}
\newcommand*{\N}{\field{N}}
\newcommand*{\R}{\field{R}}
\newcommand*{\BS}{B$\mathbf{S}$}
\newcommand{\setst}[2]{\{#1\ |\ #2\}}
\newcommand*{\powerset}[1]{\mathcal P(#1)}
\newcommand*{\powersetf}[1]{\mathcal P_{\textnormal{f}}(#1)}
\newcommand*{\powersetcof}[1]{\mathcal P_{\textnormal{cof}}(#1)}
%
%
\title{Wreath products of groups acting with bounded orbits}
\author{Paul-Henry Leemann\thanks{Supported by grant 200021\textunderscore188578 of the Swiss National Fund for Scientific Research.}, Grégoire Schneeberger}
\date{\today \quad \currenttime}
%
%
%
%
%
%
%
%
%
%
\begin{document}
\maketitle
%
%
%
%
%
%
%
%
%
%
\begin{abstract}
If S is a structure over metric spaces, we say that a group G has property \BS{} if any action on a $\mathbf S$-space has bounded orbits. Examples of such structures include metric spaces, Hilbert spaces, CAT(0) cube complexes, connected median graphs, trees or ultra-metric spaces.
They correspond respectively to Bergman's property, property FH (which, for countable groups, is equivalent to the celebrated Kazdhan's property (T)), property FW (both for CAT(0) cube complexes and for connected median graphs), property FA and cof~$\neq\omega$.

Our main result is that for a large class of structures S, the wreath product $G\wr_XH$ has property \BS{} if and only if both $G$ and $H$ have property \BS{} and $X$ is finite. On one hand, this encompasses in a general setting previously known results for properties FH and FW. On the other hand, this also apply to the Bergman property.
Finally, we also obtain that $G\wr_XH$ has cof~$\neq\omega$ if and only if both $G$ and $H$ have cof~$\neq\omega$ and $H$ acts on $X$ with finitely many orbits.
\end{abstract}
%
%
%
%
%
%
%
%
%
%
%
%
%
%
%
%
%
%
%
%
%
%
%
%
%
%%%%%%%%%%%%%%%%%%%%%%%%%%%%%%%%%%%%%%%%%%%%%%%%%%%%%%%%%%%%%%%%%%%%%%%%%%%
%%%%%%%%%%%%%%%%%%%%%%%%%%%%%%%%%%%%%%%%%%%%%%%%%%%%%%%%%%%%%%%%%%%%%%%%%%%
%%%%%%%%%%%%%%%%%%%%%%%    Section : Introduction    %%%%%%%%%%%%%%%%%%%%%%
%%%%%%%%%%%%%%%%%%%%%%%%%%%%%%%%%%%%%%%%%%%%%%%%%%%%%%%%%%%%%%%%%%%%%%%%%%%
%%%%%%%%%%%%%%%%%%%%%%%%%%%%%%%%%%%%%%%%%%%%%%%%%%%%%%%%%%%%%%%%%%%%%%%%%%%
\section{Introduction}
%
%
%
%
%
%
When working with group properties, it is natural to ask if they are stable under ``natural'' group operations. One such operation, of great use in geometric group theory, is the wreath product, which generalizes the direct product of two groups, see Section~\ref{Section:Def} for all the relevant definitions.

In the context of properties defined by fixed points of actions, the first result concerning wreath products is due to Cherix, Martin and Valette and later refined by Neuhauser and concerns property (T).
%
%
\begin{thm}[\cite{Cherix2004,Neuhauser2005a}] \label{T:Wreath_prop_T}
Let $G$ and $H$ be two discrete groups with $G$ non-trivial and let $X$ be a set on which $H$ acts. The wreath product $G \wr_X H$ has property (T) if and only if $G$ and $H$ have property (T) and $X$ is finite.
\end{thm}
%
%
For countable groups (and more generally for $\sigma$-compact locally-compact topological groups), property (T) is equivalent, by the Delorme-Guichardet's Theorem, to property FH\todo{put link to def}, see \cite[Thm. 2.12.4]{Bekka2008}.
Hence, Theorem~\ref{T:Wreath_prop_T} can also be viewed, for countable groups, as a result on property FH.

The corresponding result for property FA is a little more convoluted and was obtained a few years later by Cornulier and Kar.
%
%
\begin{thm}[\cite{Cornulier2011}]\label{Thm:FACK}
Let $G$ and $H$ be two groups with $G$ non-trivial and $X$ a set on which $H$ acts with finitely many orbits and without fixed points.
Then $G\wr_XH$ has property FA if and only if $H$ has property FA, $G/[G:G]$ is finite and $G$ has cof~$\neq\omega$.
\end{thm}
%
%
Finally, in a recent note, the authors proved an analogous of Theorem~\ref{T:Wreath_prop_T} for property FW:
%
%
\begin{thm}[\cite{LS2020}]\label{Thm:PropFW}
Let $G$ and $H$ be two groups with $G$ non-trivial and let $X$ be a set on which $H$ acts. Suppose that all three of $G$, $H$ and $G\wr_XH$ are finitely generated. Then the wreath product $G \wr_X H$ has property FW if and only if $G$ and $H$ have property FW and $X$ is finite.
\end{thm}
%
%
Since the publication of Theorem~\ref{Thm:PropFW}, Y. Stalder let us know (private communication) that the arguments of \cite{LS2020} can be adapted to spaces with walls in order to replace the finite generation hypothesis of Theorem~\ref{Thm:PropFW} by the fact that all three of $G$, $H$ and $G\wr_XH$ are at most countable.
On the other hand, A. Genevois published a new version of \cite{2017arXiv170500834G} to make explicit the fact that his ``diadem product'' construction implies that $G\wr_HH$ has property FW if and only if $G$ has property FW and $H$ is finite.

The above results on property FH, FW and FA were obtained with distinct methods even if the final results share a common flavor.
On the other hand, all three of properties FH, FW and FA can be characterized by the fact that any isometric action on a suitable metric space (respectively Hilbert space, connected median graph and tree) has bounded orbits.
But more group properties can be characterized in terms of actions with bounded orbits. This is for example the case of the Bergman's property (actions on metric spaces) or of cof~$\neq\omega$ (actions on ultrametric spaces).

By adopting the point of view of actions with bounded orbits, we obtain an unified proof of the following result; see also Theorem~\ref{Thm:Technic} for the general (and more technical) statement.
%
%
\begin{thm}\label{Thm:Main}
Let \BS{} be any one of the following properties: Bergman's property, property FH or property FW.
Let $G$ and $H$ be two groups with $G$ non-trivial and let $X$ be a set on which $H$ acts. Then the wreath product $G \wr_X H$ has property \BS{} if and only if $G$ and $H$ have property \BS{} and $X$ is finite.
\end{thm}
%
%
With a little twist, we also obtain a similar result for groups with cof~$\neq\omega$:
\begin{thm}\label{Thm:UncCoun}
Let $G$ and $H$ be two groups with $G$ non-trivial and let $X$ be a set on which $H$ acts. Then the wreath product $G \wr_X H$ has cof~$\neq\omega$ if and only if $G$ and $H$ have cof~$\neq\omega$ and $H$ acts on $X$ with finitely many orbits.
\end{thm}
%
%
A crucial ingredient of our proofs, is that the spaces under consideration admit a natural notion of Cartesian product.
In particular, some of our results do not work for trees and property FA.
Nevertheless, we are still able to show that if $G\wr_XH$ has property FA, then $H$ acts on $X$ with finitely many orbits. Combining this with Theorem~\ref{Thm:FACK} we obtain
\begin{thm}\label{Thm:FAFiniteOrbits}
Let $G$ and $H$ be two groups with $G$ non-trivial and $X$ a set on which $H$ acts. Suppose that $H$ acts on $X$ without fixed points.
Then $G\wr_XH$ has property FA if and only if $H$ has property FA, $H$ acts on $X$ with finitely many orbits, $G/[G:G]$ is finite and $G$ has cof~$\neq\omega$.
\end{thm}
%
%
%
%
\paragraph{Organization of the paper}
The next section contains all the definitions as well as some examples. Section \ref{Section:Proof} is devoted to the proof of Theorems~\ref{Thm:Main} and \ref{Thm:UncCoun} as well as to some related results.
%
%
%
\paragraph{Acknowledgment}
The authors are thankful to A. Genevois and T. Nagnibeda for helpful comments on a previous version of this note and to the NSF for its support.
%
%
%
%
%
%
%
%
%
%
%
%
%
%%%%%%%%%%%%%%%%%%%%%%%%%%%%%%%%%%%%%%%%%%%%%%%%%%%%%%%%%%%%%%%%%%%%%%%%%%%
%%%%%%%%%%%%%%%%%%%%%%%%%%%%%%%%%%%%%%%%%%%%%%%%%%%%%%%%%%%%%%%%%%%%%%%%%%%
%%%%%%%%%%%%%%%%%    Section : Definitions and examples    %%%%%%%%%%%%%%%%
%%%%%%%%%%%%%%%%%%%%%%%%%%%%%%%%%%%%%%%%%%%%%%%%%%%%%%%%%%%%%%%%%%%%%%%%%%%
%%%%%%%%%%%%%%%%%%%%%%%%%%%%%%%%%%%%%%%%%%%%%%%%%%%%%%%%%%%%%%%%%%%%%%%%%%%
\section{Definitions and examples}\label{Section:Def}
This section contains all the definitions, as well as some useful preliminary facts and some examples.
%
%
%
%
%
%
%
%
%
%
%%%%%%%%%%%%%%%%%%%%%%%%%%%%%%%%%%%%%%%%%%%%%%%%%%%%%%%%%%%%%%%%%%%%%%%%%%%
%%%%%%%%%%%%%%%%%%%%%%%%%%%%%%%%%%%%%%%%%%%%%%%%%%%%%%%%%%%%%%%%%%%%%%%%%%%
%%%%%%%%%%%%%%%%%%%%    Subsection : Wreath products    %%%%%%%%%%%%%%%%%%%
%%%%%%%%%%%%%%%%%%%%%%%%%%%%%%%%%%%%%%%%%%%%%%%%%%%%%%%%%%%%%%%%%%%%%%%%%%%
%%%%%%%%%%%%%%%%%%%%%%%%%%%%%%%%%%%%%%%%%%%%%%%%%%%%%%%%%%%%%%%%%%%%%%%%%%%
\subsection{Wreath products}
%
%
%
%
%
Let $X$ be a set and $G$ a group. We view
$\bigoplus_XG$ as the set of functions from $X$ to $G$ with finite support:
\[
	\bigoplus_XG=\setst{\varphi\colon X\to G}{\varphi(x)=1 \textnormal{ for all but finitely many }x}.
\]
This is naturally a group, where multiplication is taken componentwise.

If $H$ is a group acting on $X$, then it naturally acts on $\bigoplus_XG$
by $(h.\varphi)(x)=\varphi(h^{-1}.x)$.
This leads to the following standard definition
\begin{defn}\label{Def:WreathProd}
Let $G$ and $H$ be groups and $X$ be a set on which $H$ acts.
The \emph{(retricted) wreath product} $G\wr_XH$ is the group $(\bigoplus_XG)\rtimes H$.
\end{defn}
For $g$ in $G$ and $x$ in $X$, we define the following analogs of Kronecker's delta functions
\begin{equation*}
\delta_x^g (y) \coloneqq
\begin{cases}
g & y = x \\
1 & y \neq x.
\end{cases}
\end{equation*}
A prominent  source of examples of wreath products are the ones of the form $G\wr_HH$, where $H$ acts on itself by left multiplication.
They are sometimes called \emph{standard wreath products} or simply \emph{wreath products}, while general $G\wr_XH$ are sometimes called \emph{permutational wreath products}.
The probably most well-known example of wreath product is the so called \emph{lamplighter group} $(\Z/2\Z)\wr_\Z\Z$.
Other (trivial) examples of wreath products are direct products $G\times H$ which correspond to wreath products over a singleton $G\wr_{\{*\}}H$.
%
%
%
%
%
%
%
%
%
%
%%%%%%%%%%%%%%%%%%%%%%%%%%%%%%%%%%%%%%%%%%%%%%%%%%%%%%%%%%%%%%%%%%%%%%%%%%%
%%%%%%%%%%%%%%%%%%%%%%%%%%%%%%%%%%%%%%%%%%%%%%%%%%%%%%%%%%%%%%%%%%%%%%%%%%%
%%%%%%%%%%%%%%    Subsection : Actions with bounded orbits    %%%%%%%%%%%%%
%%%%%%%%%%%%%%%%%%%%%%%%%%%%%%%%%%%%%%%%%%%%%%%%%%%%%%%%%%%%%%%%%%%%%%%%%%%
%%%%%%%%%%%%%%%%%%%%%%%%%%%%%%%%%%%%%%%%%%%%%%%%%%%%%%%%%%%%%%%%%%%%%%%%%%%
\subsection{Actions with bounded orbits}
%
%
%
Remind that a metric space $(X,d)$ is \emph{ultrametric} if $d$ satisfies the strong triangular inequality: $d(x,y)\leq\max\{d(x,z),d(z,y)\}$ for any $x$, $y$ and $z$ in $X$. 

For $u$ and $v$ two vertices of a connected\footnote{We will always assume that our connected graphs are non-empty. This is coherent with the definition that a connected graph is a graph with exactly one connected component.} graph $\mathcal G$, we define the total interval $[u,v]$ as the set of vertices that lie on some shortest path between $u$ and $v$.
A connected graph $\mathcal G$ is \emph{median} if for any three vertices $u$, $v$, $w$, the intersection $[u,v]\cap[v,w]\cap[u,w]$ consists of a unique vertex, denoted $m(u,v,w)$.
A graph is \emph{median} if each of its connected component is median. For more on median graphs and spaces see \cite{MR2405677,MR1705337,MR2671183}.
%
%
%
%
\paragraph{Some group properties}
We now remind the definition of various properties for actions on metric spaces.\PH{Peut-être mettre une phrase avec les réf de première fois où ça été étudié. Soit avant la déf, soit dans le paragraphe après}
%
%
\begin{defn}\label{Def:FHFA}
Let $G$ be a group.
It is said to have
\begin{itemize}
\item\emph{property SB} if any action on a metric space has bounded orbits,
\item \emph{property FH} if any action on a real Hilbert space has bounded orbits,
\item
\emph{property FW} if any action on a connected median graph has bounded orbits,
\item
\emph{property FA} if any action on a tree has bounded orbits,
\item
\emph{cof~$\neq\omega$} if any action on an ultrametric space has bounded orbits.
\end{itemize}
In the above, \emph{actions} are supposed to preserve the structure. In particular, actions on (ultra)metric spaces are by isometries, actions on graphs (including trees) are by graph isomorphisms and actions on Hilbert spaces are by scalar product preserving isometries.
\end{defn}
%
%
The names FH, FW and FA come from the fact that these properties admit a description in terms of (and were fist studied in the context of) existence of a Fixed point for actions on Hilbert spaces, on spaces with Walls (or equivalently on CAT(0) cube complexes) and on trees (\emph{Arbres} in french).
On the other hand, SB stands for Strongly Bounded and is also called the \emph{Bergman's property}.
Finally, a group has cof~$\neq\omega$ (does not have \emph{countable cofinality}) if and only if it cannot be written has a countable increasing union of proper subgroups, see Lemma~\ref{Lemma:CofSub}.

For countable groups (and more generally for $\sigma$-compact locally compact groups), property FH is equivalent to the celebrated Kazdhan's property (T) by the Delorme-Guichardet theorem, but this is not true in general. Indeed in \cite{MR2240370}, Cornulier constructed an uncountable discrete group $G$ with property SB which, as we will see just below, implies property FW. Such a group cannot have property (T) as, for discrete groups, it implies finite generation.

We have the following strict implications between the properties of Definition~\ref{Def:FHFA} %\cite{MR1432323,MR0476875,MR3299841,Cornulier2013}
\begin{equation}\tag{\dag}\label{EQ:Implications}
	\textnormal{SB}\implies \textnormal{FH}\implies \textnormal{FW}\implies \textnormal{FA}\implies\textnormal{cof~$\neq\omega$}.
\end{equation}
The implications $\textnormal{SB}\implies \textnormal{FH}$ and $\textnormal{FW}\implies \textnormal{FA}$ trivially follows from the fact that Hilbert spaces are metric space and trees that trees are connected median graphs.
The implication $\textnormal{FA}\implies\textnormal{cof~}\neq\omega$ is due to Serre \cite{MR0476875}: if $G$ is an increasing union of subgroups $G_i$, then $\bigsqcup G/G_i$ admits a tree structure by joining any $gG_i\in G/G_i$ to $gG_{i+1}\in G/G_{i+1}$. The action of $G$ by multiplication on $\bigsqcup G/G_i$ is by graph isomorphisms and with unbounded orbits.
Finally, the implication $\textnormal{FH}\implies \textnormal{FW}$ follows from the fact that a group $G$ has property FW if and only if any action on a real Hilbert space which preserves integral points has bounded orbits \cite{Cornulier2013}.

On the other hand, here are some examples for the strictness of the implications of \eqref{EQ:Implications}.
An infinite finitely generated group with property (T), e.g. $\SL_3(\Z)$, has property FH, but it does not have property SB since its action on its Cayley graph is transitive and hence has an unbounded orbit.
The group $\SL_2(\Z[\sqrt{2}])$ has property FW but not FH, see \cite{MR3299841}.
If $G$ is a non-trivial finite group and $H$ is an infinite group with property FA, then $G\wr_HH$ has property FA by Theorem~\ref{Thm:FAFiniteOrbits}, but does not have property FW by Theorem~\ref{Thm:Main}.
Finally, $\Z$ has cof~$\neq\omega$, while it acts by translations and with unbounded orbits on the infinite $2$-regular tree.

It is possible to consider relative version of the properties appearing in Definition~\ref{Def:FHFA}.\PH{J'ai mis cela ici, mais on peut y bouger dans la prochaine sous-section.}
If $G$ is a group, $H$ a subgroup of $G$ and BS the property of Definition~\ref{Def:FHFA} defined by ``every action on S-spaces is bounded'', we say that the pair $(G,H)$ has \emph{relative property BS} if for every $G$ action on a S-space, the $H$ orbits are bounded.
A group $G$ has property BS if and only if for every subgroup $H$ the pair $(G,H)$ has relative property BS, if and only if fo every overgroup $L$ the pair $(L,G)$ has relative property BS.
%
%
%
%
\paragraph{More on median graphs}
Trees are the simplest examples of median graphs and a simple verification shows that if $X$ and $Y$ are both (connected) median graphs, then their cartesian product is also a (connected) median graph.
On the other hand, the following example will be fundamental for us.
%
%
\begin{exmp}\label{Ex:MainMedian}
Let $X$ be a set and let  $\powerset{X}=2^X$ be the set of all subsets of~$X$.
Define a graph structure on $\powerset{X}$ by putting an edge between $E$ and $F$ if and only if $\#(E\Delta F)=1$, where $\Delta$ is the symmetric difference.
Therefore, the distance between two subsets $E$ and $F$ is $E\Delta F$ and
the connected component of $E$ is the set of all subsets $F$ with $E\Delta F$ finite.
For $E$ and $F$ in the same connected component, $[E,F]$ consist of all subsets of $X$ that both contain $E\cap F$ and are contained in $E\cup F$.
In particular, $\powerset{X}$ is a median graph, with $m(D,E,F)$ being the set of all elements belonging to at least two of $D$, $E$ and $F$. In other words, $m(D,E,F)=(D\cap E)\cup(D\cap F)\cup(E\cap F)$.
\end{exmp}
%
%
We denote by $\powersetf{X}$, respectively $\powersetcof{X}$ the set of all finite, respectively cofinite, subsets of $X$.
They are connected components of $\powerset{X}$, which coincide if and only if $X$ is finite.
More generally, the connected components of $\powerset{X}$ are hypercubes and it turns out that every connected median graph is a retract of a connected component of some $\powerset{X}$, see \cite{Bandelt1984}.

These graphs will be fundamental for us due to the following fact.
Any action of a group $G$ on a set $X$ naturally extends to an action of $G$ on $\powerset{X}$ by graph homomorphisms: $g.\{x_1,\dots,x_n\}=\{g.x_1,\dots,g.x_n\}$.
Be careful that the action of $G$ on $\powerset{X}$ may exchange the connected components.
In fact, the connected component of $E\subset X$ is stabilized by $G$ if and only if $E$ is \emph{commensurated} by~$G$, that is if for every $g\in G$ the set $E\Delta gE$ is finite.
For example, both $\powersetf{X}$ and $\powersetcof{X}$ are always preserved by the action of $G$.

\paragraph{Groups with cof~$\neq\omega$}
The following characterization of groups which have not cof~$\neq\omega$ is well-known and we include a proof only for the sake of complexity.
It implies in particular that a countable group has cof~$\neq\omega$ if and only if it is finitely generated.
%
%
\begin{lem}\label{Lemma:CofSub}
Let $G$ be a group. Then the following are equivalent:
\begin{enumerate}
\item $G$ can be written as a countable increasing union of proper subgroups,
\item $G$ does not have cof~$\neq\omega$, i.e. there exists an ultrametric space $X$ on which $G$ acts with an unbounded orbit,
\item There exists a $G$-invariant (for the action by left multiplication) ultrametric $d$ on $G$ such that $G\curvearrowright G$ has an unbounded orbit.
\end{enumerate}
\end{lem}
\begin{proof}
It is clear that the third item implies the second.

Suppose that $(X,d)$ is an ultrametric space on which $G$ acts with an unbounded orbit and let $x_0$ be an element of $X$ such that $G.x_0$ is unbounded. For any $n\in \N$ let $H_n$ be the subset of $G$ defined by
\[
	H_n\coloneqq\setst{g\in G}{d(x_0,g.x_0)\leq n}.
\]
It is clear that $G$ is the increasing union of the (countably many) $H_n$.
On the other hand, the $H_n$ are subgroups of $G$. Indeed, $H_n$ is trivially closed under taking the inverse, and is also closed under taking products since $d(x_0,gh.x_0)\leq\max\{d(x_0,g.x_0),d(g.x_0,gh.x_0)\}=\max\{d(x_0,g.x_0),d(x_0,h.x_0)\}$. As $G.x_0$ is unbounded, they are proper subgroups.

Finally, suppose that $G=\bigcup_{n\in \N}H_n$ where the $H_n$ form an increasing sequence of proper subgroups. It is always possible to suppose that $H_0=\{1\}$.
Define $d$ on $G$ by $d(g,h)\coloneqq\min\setst{n}{g^{-1}h\in H_n}$.
One easily verifies that $d$ is a $G$-invariant ultrametric. Moreover, the orbit of $1$ contains all of $G$ and is hence unbounded.
\end{proof}
%
%
A slight variation of the above lemma gives us
\begin{exmp}\label{Exmpl:Ultra}
Let $(G_i)_{i\geq 1}$ be non-trivial groups and let $G\coloneqq\bigoplus_{i\geq 1} G_i$ be their direct sum.
Then $d_\infty(f,g)\coloneqq\max\setst{i}{f(i)\neq g(i)}$ is an ultrametric on $G$ which is $G$-invariant for the action of $G$ on itself by left multiplication.
\end{exmp}
%
%
%
%
%
%
%
%
%
%
%%%%%%%%%%%%%%%%%%%%%%%%%%%%%%%%%%%%%%%%%%%%%%%%%%%%%%%%%%%%%%%%%%%%%%%%%%%
%%%%%%%%%%%%%%%%%%%%%%%%%%%%%%%%%%%%%%%%%%%%%%%%%%%%%%%%%%%%%%%%%%%%%%%%%%%
%%%%%%%%%%%    Groups acting with bounded orbits on $\mathbf S$-spaces    %%%%%%%%%%%
%%%%%%%%%%%%%%%%%%%%%%%%%%%%%%%%%%%%%%%%%%%%%%%%%%%%%%%%%%%%%%%%%%%%%%%%%%%
%%%%%%%%%%%%%%%%%%%%%%%%%%%%%%%%%%%%%%%%%%%%%%%%%%%%%%%%%%%%%%%%%%%%%%%%%%%
\subsection{Groups acting with bounded orbits on \texorpdfstring{$\mathbf S$}{\textbf{S}}-spaces}
It is possible to define other properties in the spirit of Definition~\ref{Def:FHFA} for any ``additional structure on metric spaces''.
In order to give a uniform treatment of all of them, we will use the notion of quasipseudo-metric spaces and the language of category theory.
A reader unfamiliar with category theory and interested only in one specific structure (as for example Banach spaces and groups acting with bounded orbits on them), might forget all this general considerations and only verify that the arguments of Section~\ref{Section:Proof} apply for their favorite structure.
%
%
\begin{defn}
A \emph{quasipseudo-metric space} is a set $X$ with a map $d\colon X\times X\to \R_{\geq0}$, called a \emph{quasipseudo-distance}, such that
\begin{enumerate}
\item $d(x,x)=0$ for all $x\in X$,
\item $d(x,z)\leq d(x,y)+d(y,z)$.
\end{enumerate}
\end{defn}
%
%
If moreover $d(x,y)=d(y,x)$ and $d(x,y)\neq 0$ for $x\neq y$, the map $d$ is a \emph{distance} and $(X,d)$ is a metric space.
On the other hand, a \emph{ultra-quasipseudo-metric space} is a quasipseudo-metric space $(X,d)$ such that $d$ satisfies the strong triangular inequality.
A \emph{morphism} (or \emph{short map}) between two quasipseudo-metric spaces $(X_1,d_1)$ and $(X_2,d_2)$ is a distance non-increasing map $f\colon X_1\to X_2$, that is $d_2(f(x),f(y))\leq d_1(x,y)$ for any $x$ and $y$ in $X_1$.
If $f$ is bijective and distance preserving, then it is an \emph{isomorphism} (or \emph{isometry}).
Quasipseudo-metric spaces with short maps form a category $\mathbf{QPMet}$ of which the category of metric spaces (with short maps) $\mathbf{Met}$ is a full subcategory.

If $(X,d)$ is a quasipseudo-metric space, we have a natural notion of the  \emph{diameter} of a subset $Y\subset X$ with value in $[0,\infty]$, defined by $\diam(Y)\coloneqq\sup\setst{d(x,y)}{x,y\in Y}$.
%
%
\begin{defn}\label{Def:Categoric}
An \emph{additional structure on quasipseudo-metric spaces}, or a \emph{qp-metric structure} for short is a concret category $(\mathbf S,F_{\mathbf S})$ over $\mathbf{QPMet}$.
That is, it is a category $\mathbf{S}$ together with a faithful functor $F_{\mathbf S}\colon \mathbf{S}\to \mathbf{QPMet}$.
The objects of $\mathbf{S}$ are called \emph{$\mathbf S$-spaces} and the morphisms \emph{$\mathbf S$-morphisms}.

A \emph{$G$-action on a $\mathbf S$-space} $X$ is simply an homomorphism $\alpha\colon G\to\Aut_{\mathbf{S}}(X)$. It has \emph{bounded orbits} if $F_{\mathbf S}\circ \alpha\colon G\to \Aut_{\mathbf{QPMet}}(X)$ has bounded orbits.
\end{defn}
%
%
In practice, we will often simply write $\mathbf{S}$  for the pair $(\mathbf{S},F_{\mathbf{S}})$.
Since $\mathbf{QPMet}$ itself is concrete, that is we have a faithful functor $F\colon \mathbf{QPMet}\to\mathbf{Set}$, the category $\mathbf{S}$ is also concrete (via $F\circ F_S$) and its objects can be thoughts as sets with ``extra structure''.

In practice, a lot of examples of concrete categories over $\mathbf{QPMet}$ factor through the category $\mathbf{Met}$.
Obvious examples of concrete categories over $\mathbf{Met}$ include metric spaces and ultrametric spaces (with short maps).
Hilbert spaces, Banach spaces and more generally (semi-) normed spaces are also concrete over $\mathbf{Met}$ if we restrict ourself to morphisms that do not increase the distance (that is such that $\langle f(x)\mid f(y)\rangle\leq\langle x\mid y\rangle$, respectively $\|f(x)\|\leq\| x\|$). In particular, for us isomorphisms of Hilbert and Banach spaces will always be isometries.
 
For connected graphs (and hence for connected median graphs and for trees), one looks at the category $\mathbf{Graph}$ where objects are connected simple graphs $G=(V,E)$ and where a morphism $f\colon (V,E)\to(V',E')$ is a function between the vertex sets such that if $(x,y)$ is an edge then either $f(x)=f(y)$ or $(f(x),f(y))$ is an edge.
The functor $F_{\mathbf S}\colon\mathbf{Graph}\to\mathbf{Met}$ sends a connected graph to its vertex set together with the graph distance: $d(x,y)$ is the minimum number of edges on a path between $x$ and $y$.

We can now, formally define the group property \BS{} as:
%
\begin{defn}\label{Def:PropBS}
Let $(\mathbf S,F_{\mathbf S})$ be a qp-metric structure.
A group $G$ has \emph{property \BS} if every $G$-action on a $\mathbf S$-space has bounded orbits.
A pair $(G,H)$ of a group and a subgroup has \emph{relative property \BS} if for every $G$-action on a $\mathbf S$-space, the $H$ orbits are bounded.
\end{defn}
%
%
All the properties of Definition~\ref{Def:FHFA} are of the form \BS.
Examples of other interesting properties include looking at $\mathbf{R}$-trees (also called \emph{real trees}) or at (some specific subclass of) Banach spaces.
The property F$\mathbf{R}$ of having bounded actions on $\mathbf{R}$-trees is known to be strictly stronger than FA \cite{MR3465847}. Theorem~\ref{Thm:FAFiniteOrbits} holds whenever property FA is replaced by property F$\mathbf{R}$, with a similar proof.
On the other hand, the property of having bounded orbits for action on Banach spaces satisfy the analog of Theorem~\ref{Thm:Main}, see Theorem~\ref{Thm:Technic}.

An example of an uninteresting property \BS{} is given by taking $\mathbf{S}$ to be category of metric spaces of bounded diameter (together with short maps). Indeed, in this case, any group has \BS.

The category $\mathbf{QPMet}$ has the advantage (over $\mathbf{Met}$) of behaving more nicely with respects to categorical constructions.
However, we have\PH{Ajouté cette phrase, le lemme et la preuve. C'est utilisé implicitement dans la preuve de Prop~\ref{Prop:Extension}.}
%
%
\begin{lem}
A group $G$ has the Bergman's Property (respectively cof~$\neq\omega$) if and only if any $G$ action on a quasipseudo-metric (respectively ultra-quasipseudo-metric) space has bounded orbits.
\end{lem}
\begin{proof}
One direction is trivial.

For the other direction, let $(X,d)$ be a quasipseudo-metric space on which $G$ acts by isometries.
Let $d'(x,y)\coloneqq\frac12(d(x,y)+s(y,x)$ be the symmetrization of $d$.
Then the action of $G$ on $X$ is by $d'$-isometries and a subset $Y\subset X$ is bounded for $d$ if and only if it is bounded for $d'$.
Finally, let $\tilde X\coloneqq X/\sim$ be the quotient of $X$ for the relation $x\sim y$ if $d'(x,y)=0$ and let $\tilde d$ be the quotient of $d'$.
Then $(\tilde X,\tilde d)$ is a metric space, the action of $G$ passes to the quotient and $G.x$ is $d'$ bounded (hence $d$ bounded) if and only if $G.[x]$ is $\tilde d$ bounded.
Finally, if $d$ satisfies the strong triangular inequality, then so does $d'$ and $\tilde d$.
\end{proof}
%
%
While we will be able to obtain some results for a general qp-metric structure~$\mathbf{S}$, we will sometimes need to restrict ourself to structure with a suitable notion of cartesian product.
%
%
\begin{defn}\label{Def:Cartesian}
A qp-metric structure $(\mathbf S,F_{\mathbf S})$ has \emph{cartesian powers} if for any $\mathbf S$-space $X$ and any integer $n$, there exists a $\mathbf S$-object, called the \emph{$n$\textsuperscript{th} cartesian power of $X$} and written $X^n$, such that:
\begin{enumerate}
\item
$X^n$ is compatible with the cartesian product of sets. That is $F\circ F_{\mathbf S}(X^n)$ is the set cartesian power.
\item\label{Condidef:2}
If $E\subset X$ is unbounded, then the diagonal $\diag(E)\subset X^n$ is unbounded.
\item\label{Item:Product}
$\Aut_{\mathbf{S}}(X)^n\rtimes \Sym(n)$ is a subgroup of $\Aut_{\mathbf{S}}(X^n)$.
\end{enumerate}
\end{defn}
%
%
For (quasipseudo-/ultra-) metric spaces, the categorical product (corresponding to the metric $d_\infty=\max\{d_X,d_Y\}$) works fine, but any product metric of the form $d_p=(d_X^p+d_Y^p)^{\frac1p}$ for $p\in[1,\infty]$ works as well.
For Hilbert and Banach spaces, we take the usual cartesian product (which is also the categorial product), which corresponds to the metric $d_2=\sqrt{d_X^2+d_Y^2}$.
For connected median graphs, the usual cartesian product (which is not the categorical product!\footnote{The categorial product in $\mathbf{Graph}$ is the strong product.}) with $d_1=d_X+d_Y$ works well.
On the other hand, trees do not have cartesian powers.
%
%
\begin{rem}
In view of Definition~\ref{Def:PropBS} and \ref{Def:Cartesian}, the reader might ask why we are working in $\mathbf{QPMet}$ instead of $\mathbf{Born}$ the category of bornological spaces together with bounded maps.
The reason behind this is the forthcoming Lemma~\ref{Lemma:InterProd} and its corollaries, which fail for general bornological spaces.\PH{Je pense qu'on peut en rester là et ne pas parler de l'exemple $\{$bornés$\}=\{$finis$\}$.}
\end{rem}
%
%
We conclude this section by a remark\PH{Peut-être à enlever de l'article, non ? Mais à garder pour ta thèse ?} on a variation of Definition~\ref{Def:FHFA}.
One might wonder what happens if in Definition~\ref{Def:FHFA} we replace the requirement of having bounded orbits by having uniformly bounded orbits.
It turns out that this is rather uninteresting as a group $G$ is trivial if and only if any $G$-action on a metric space (respectively on an Hilbert space, on a connected median graph, on a tree or on an ultrametric space) has uniformly bounded orbits.
Indeed, if $G$ is non-trivial, then for the action of $G$ on the Hilbert space $\ell^2(G)$ the orbit of $n\cdot \delta_g$ has diameter $n\sqrt2$.
For a tree (and hence also for a connetected median graph), one may look at the tree $T$ obtained by taking a root $r$ on which we glue an infinite ray for each elements of $G$. Then $G$ naturally acts on $T$ by permuting the rays. The orbits for this action are the $\mathcal L_n=\setst{v}{d(v,r)=n}$ which have diameter $2n$.
Finally, it is possible to put an ultradistance on the vertices of $T$ by  $d_\infty(x,y)\coloneqq\max\{d(x,r),d(y,r)\}$ if $x\neq y$. Then the orbits are still the~$\mathcal L_n$, but this time with diameter $n$.
%
%
%
%
%
%
%
%
%
%
%%%%%%%%%%%%%%%%%%%%%%%%%%%%%%%%%%%%%%%%%%%%%%%%%%%%%%%%%%%%%%%%%%%%%%%%%%%
%%%%%%%%%%%%%%%%%%%%%%%%%%%%%%%%%%%%%%%%%%%%%%%%%%%%%%%%%%%%%%%%%%%%%%%%%%%
%%%%%%%%%%%%%%%%%    Section : Proof of the main result    %%%%%%%%%%%%%%%%
%%%%%%%%%%%%%%%%%%%%%%%%%%%%%%%%%%%%%%%%%%%%%%%%%%%%%%%%%%%%%%%%%%%%%%%%%%%
%%%%%%%%%%%%%%%%%%%%%%%%%%%%%%%%%%%%%%%%%%%%%%%%%%%%%%%%%%%%%%%%%%%%%%%%%%%
\section{Proofs of the main results}
\label{Section:Proof}
%
%
%
Throughout this section, $\mathbf{S}$ will denote a qp-metric structure and \BS{} the group property ``every action on a $\mathbf{S}$-space has bounded orbits''.
Heuristically, a $\mathbf{S}$-space is a (quasipseudo-) metric space with an additional structure (as for example an Hilbert space). For a precise definition, see Definition~\ref{Def:PropBS}.

We begin this section with two easy but useful results.
%
%
\begin{lem}\label{Lemma:Quotient}
Let $G$ be a group and $H$ be a quotient.
If $G$ has property \BS, then so has $H$.
\end{lem}
\begin{proof}
We have $H\cong G/N$. If $H$ acts on some $\mathbf{S}$-space $X$ with an unbounded orbit, then the $G$ action on $X$ defined by $g.x\coloneqq gN.x$ has also an unbounded orbit.
\end{proof}
%
%
\begin{lem}\label{Lemma:InterProd}
Let $G$ be a group and $A$ an $B$ be two subgroups such that $G=AB$.
If both $(G,A)$ and $(G,B)$ have relative property \BS, then $G$ has property \BS.
\end{lem}
\begin{proof}
Let $X$ be a $\mathbf{S}$-space and $x$ be an element of $X$.
Let $D_1$ be the diameter of $B.x$ and $D_2$ the diameter of $A.x$. By assumption, they both are finite.
Since $A$ acts by isometries, all the $a.Bx$ have diameter $D_1$.
Let $y$ be an element of $G.x$. There exists $a\in A$ such that $y$ belongs to $a.Bx$. Since $1$ belongs to $B$, $y$ is at distance at most $D_1$ of $a.x$. Altogether we have that the diameter of $G.x$ is bounded above by $2D_1+D_2$.
\end{proof}
%
%
By combining Lemmas~\ref{Lemma:Quotient} and~\ref{Lemma:InterProd}, we obtain the following three corollaries on direct, semi-direct and wreath products.\PH{J'ai mis trois corollaires disctinct car cela me semblait moins indigeste. Tu trouveras dans le fichier TeX la version avec un seul gros corollaire.}
%\PH{J'ai tout mis en un corollaire, mais on peut choisir de séparer, ou de ne pas tout dire comme certains points sont redondants. Les ``anciens résultats sont dans le fichier TeX''}
%
%
%\begin{cor}\label{}
%Let $G$ and $H$ be two groups and $X$ a set on which $H$ acts. Then
%\begin{enumerate}
%\item
%$G\times H$ has property \BS{} if and only if both $G$ and $H$ have property \BS,
%\item 
%If $G\rtimes H$ has property \BS, then so has $H$,
%\item
%If both $N$ and $H$ have property \BS, then $N\rtimes H$ also has property \BS,
%\item
%If $G\wr_X H$ has property \BS, then so has $H$,
%\item
%If both $G$ and $H$ have property \BS{} and $X$ is finite, then $G\wr_X H$ has property~\BS.
%\end{enumerate}
%\end{cor}
%
%
%We also have the following lemma on semi-direct products:
%
%
\begin{cor}\label{Cor:Prod}
Let $G$ and $H$ be two groups. Then $G\times H$ has property \BS{} if and only if both $G$ and $H$ have property \BS.
\end{cor}
%
%
\begin{cor}\label{Cor:Semidirect}
Let $N\rtimes H$ be a semidirect product. Then
\begin{enumerate}
\item
If $N\rtimes H$ has property \BS, then so has $H$.
\item
If both $N$ and $H$ have property \BS, then $N\rtimes H$ also has property \BS.
\end{enumerate}
\end{cor}
%\begin{proof}
%The first part is Lemma~\ref{Lemma:Quotient}.
%
%On the other hand, suppose that $N$ and $H$ have \BS{} and let $X$ be a $\mathbf S$-space on which $G$ acts.
%Then both $N$ and $H$ acts on $X$ with bounded orbits.
%Let $x$ be an element of $X$, $D_1$ be the diameter of $H.x$ and $D_2$ be the diameter of $N.x$.
%Since $G$ acts by isometries, for every $h$ in $H$ the set $N.(h.x)=Nh.x=hN.x=h.(N.x)$ has also diameter $D_2$.
%Therefore, every element of $G.x=NH.x$ is at distance at most $D_1+D_2$ of $x$, which implies that the orbit $G.x$ is bounded.
%\end{proof}
%%
%%
%As a direct corollary, we have
%\begin{cor}\label{Cor:Prod}
%Let $G$ and $H$ be two groups. Then $G\times H$ has property \BS{} if and only if both $G$ and $H$ have property \BS.
%\end{cor}
%%
%%
%By iterating Lemma~\ref{Lemma:Semidirect}, we obtain
%%
%%
\begin{cor}\label{Cor:Wreath}
Let $G$ and $H$ be two groups and $X$ a set on which $H$ acts. Then,
\begin{enumerate}
\item
If $G\wr_X H$ has property \BS, then so has $H$,
\item
If both $G$ and $H$ have property \BS{} and $X$ is finite, then $G\wr_X H$ has property~\BS.
\end{enumerate}
\end{cor}
%
%
When $\mathbf{S}$ has a suitable notion of quotients (by a group of isometries), it is possible to obtain a strong version of Lemmas~\ref{Lemma:Quotient} and~\ref{Lemma:InterProd}.
Here is the corresponding result for Bergman's property and cof~$\neq\omega$.\PH{Je n'ai pas mis l'énoncé général, car les autres espaces qui nous intéressent ne sont pas stable par quotients.}
%
%
\begin{prop}\label{Prop:Extension}
Let \BS{} be either the Bergman's property, or the property cof~$\neq\omega$.
Let $1\to N\to G\to H\to 1$ be a group extension.
Then $G$ has property \BS{} if and only if $H$ has property \BS{} and the pair $(G,N)$ has the relative \BS{} property.
\end{prop}
\begin{proof}
One direction is simply Lemma~\ref{Lemma:Quotient} and the definition of relative Bergman's property.

On the other hand, let $(X,d)$ be a quasipseudo-metric space on which $G$ acts by isometries and let $x$ be an element of $X$.
Let $\setst{g_i}{i\in I}$ a transversal for $N$, that is $H\cong\{g_iN\}$.
By assumption, $N.x$ is bounded of diameter $D_1$ and for any $i\in I$ the subset $g_iN.x$ of $X$ has also diameter $D_1$.
Since $N$ is a subgroup of isometries of $X$, the map $d'\colon X/N\times X/N\to\R$ defined by $d'([x],[y])\coloneqq\inf\setst{d(x',y')}{x'\in N.x,y'\in N.y}$ is the quotient quasipseudo-distance\PH{En général, pour $X/\sim$, $d'$ ne satisfait pas l'inégalité triangulaire, mais c'est vrai pour $X/N$. Est-ce qu'il faut expliciter ?} on $X/N$.
Moreover, if $d$ satisfied the strong triangle inequality, then so does $d'$.
The quotient action of $H\cong G/N$ on $X/N$ is by isometries and the diameter of $H.xN$ is bounded, say by $D_2$.
In particular, for any $i$ and $j$ in $I$, the distance between the subsets $g_iN.x$ and $g_jN.x$ of $X$ is bounded by $D_2$.
Since this distance is an infimum, there exists actual elements of $g_iN.x$ and $g_jN.x$ at distance less than $D_2+1$.
Altogether, we obtain that the diameter of $G.x$ is bounded by $2D_1+D_2+1$.
\end{proof}
%
%
Since the square graph, which is not median, is a quotient of the $2$-regular infinite tree by a subgroup of isometries, the proof of Proposition~\ref{Prop:Extension} does not carry over for properties FW and FA.
However, the statement of Proposition~\ref{Prop:Extension} (stability under extension) remains true for this properties by \cite{Cornulier2013} and \cite{MR0476875}.\PH{Ce serait peut-être bien de regarder ce que l'on connait pour la propriété FH; n.b. (T) est stable par extensions.}

We now state a result on infinite direct sums.
%
%
\begin{lem}\label{Lemma:Cof}
Suppose that \BS{} implies cof~$\neq\omega$. Then
\begin{enumerate}
\item An infinite direct sum of non-trivial groups does not have \BS,
\item If $G\neq\{1\}$, then $\bigoplus_XG$ has \BS{} if and only if $G$ has \BS{} and $X$ is finite,
\item If $G\wr_XH$ has \BS, then $H$ acts on $X$ with finitely many orbits.
\end{enumerate}
\end{lem}
%
%
It is of course possible to prove Lemma~\ref{Lemma:Cof} using the characterization of cof~$\neq\omega$ in terms of subgroups. However, we find enlightening to prove it using the characterization in terms of actions on ultrametric spaces.
%
%
\begin{proof}[Proof of Lemma~\ref{Lemma:Cof}.]
By Corollary~\ref{Cor:Prod}, it is enough to prove the first assertion for countable direct sums of groups. Indeed, if $X$ is infinite, there exists a countable subset $Y \subset X$. Let $Z \coloneqq X\setminus Y$, thus we have $X = Y \sqcup Z$. We can decompose the direct sum as $\bigoplus_XG = (\bigoplus_YG) \times (\bigoplus_ZG)$ and then, by Corollary~\ref{Cor:Prod}, if $\bigoplus_YG$ does not have cof~$\neq\omega$, then neither does $\bigoplus_XG$.
So let $G\coloneqq \bigoplus_{i\geq 1}G_i$ and for each $i$, choose $g_i\neq 1$ in $G_i$.
Let $d_\infty(f,g)\coloneqq\max\setst{i}{f(i)\neq g(i)}$ be the $G$-invariant ultrametric of Example~\ref{Exmpl:Ultra}.
Then for every integer $n$, the orbit $G.1_G$ contains $\{g_1,\dots,g_n,1,\dots\}$ which is at distance $n$ of $1_G$ for $d_\infty$ if the $g_i$ are not equal to $1$.
In particular, an infinite direct sum of non-trivial groups does not have cof~$\neq\omega$, nor does it have \BS.
The second assertion follows of the first assertion combined with Corollary~\ref{Cor:Prod}.

The last assertion is a simple variation on the first.
Indeed, we have
\[
	G\wr_XH\cong(\bigoplus_{Y\in X/H}L_Y)\rtimes H\qquad\textnormal{with}\qquad L_Y\cong\bigoplus_{y\in Y}G_y,
\]
where $X/H$ is the set of $H$-orbits.
The important fact for us is that $H$ fixes the decomposition into $L_Y$ factors: for all $Y$ we have $H.L_Y=L_Y$.
Up to regrouping some of the $L_Y$ together we hence have $G\wr_XH\cong\bigl(\bigoplus_{i\geq 1}L_i\bigr)\rtimes H$ with $H.L_i=L_i$ for all $i$.
Now, we have an ultradistance $d_\infty$ on $L\coloneqq\bigoplus_{i\geq 1}L_i$ as above and we can put the discrete distance $d$ on $H$.
Then $d_\infty'=\max\{d_\infty,d\}$ is an ultradistance on $\bigl(\bigoplus_{i\geq 1}L_i\bigr)\rtimes H$, which is $\bigl(\bigoplus_{i\geq 1}L_i\bigr)\rtimes H$-invariant (for the action by left multiplication).
From a practical point of view, we have $d'_\infty\bigl((\varphi,h),(\varphi',h')\bigr)\coloneqq\max\setst{i}{\varphi(i)\neq \varphi'(i)}$ if $\varphi\neq \varphi'$ and $d'_\infty\bigl((\varphi,h),(\varphi,h')=1$ if $h\neq h'$
Since the action of $L$ on itself has an unbouded orbit for $d_\infty$, the action of $\bigl(\bigoplus_{i\geq 1}L_i\bigr)\rtimes H$ on itself has an unbounded orbit for $d'_\infty$.
\end{proof}
%
%
While the statement (and the proof) of Lemma~\ref{Lemma:Cof} is expressed in terms of cof~$\neq\omega$, it is also possible to state it and prove it for a qp-metric structure $\mathbf{S}$ without a priori knowing if \BS{} is stronger than cof~$\neq\omega$.
The main idea is to find a ``natural'' $\mathbf S$-space on which $G=\bigoplus_{i\geq 1}G_i$ acts. For example, for Hilbert spaces, one can take $\bigoplus_{i\geq 1}\ell^2(G_i)$. For connected median graphs, one takes the connected component of $\{1_{G_1},1_{G_2},\dots\}$ in $\powerset{\bigsqcup_{i\geq 1} G_i}$.
For trees, it is possible to put a forest structure on $\powerset{\bigsqcup_{i\geq 1} G_i}$ in the following way.
For $E\in\powerset{\bigsqcup_{i\geq 1} G_i}$, and for each $i$ such that $E\cap G_j$ is empty for all $j\leq i$, add an edge from $E$ to $E\cup\{g\}$ for each $g\in G_i$. The graph obtained this way is a $G$-invariant subforest of the median graph on $\powerset{\bigsqcup_{i\geq 1} G_i}$.
%
%
\begin{lem}\label{Lemma:XFinite}
Suppose that \BS{} implies FW.
Let $G$ and $H$ be two groups with $G$ non-trivial and let $X$ be a set on which $H$ acts.
If $G\wr_XH$ has \BS, then $X$ is finite.
\end{lem}
\begin{proof}
We will prove that if $X$ is infinte, then $G\wr_XH$ does not have property FW. Suppose that $X$ is infinite.
The group $\bigoplus_XG$ acts coordinatewise on  $\bigsqcup_XG$: the group $G_x$ acting by left multiplication on $G_x$ and trivially on $G_y$ for $y\neq x$. On the other hand, $H$ acts on $\bigsqcup_XG$ by permutation of the factors.
Altogether we have an action of $G\wr_XH$ on $\bigsqcup_XG$ and hence on the median graph $\powerset{\bigsqcup_XG}$.
Let $\mathbf 1\coloneqq\bigcup_{x\in X} 1_{G}$ be the subset of $\powerset{\bigsqcup_XG}$ consisting of the identity elements of all the copies of $G$.
Since every element of $\bigoplus_XG$ has only a finite number of non-trivial coordinates, the action of $G\wr_XH$ preserves the connected components of $\mathbf 1$ (and in fact every connected component of $\powerset{\bigsqcup_XG}$).

Let $I=\{i_1,i_2,\dots\}$ be a countable subset of $X$ and for every $i\in I$, choose a non-trivial $g_i\in G_{i}$.
Then the orbit of the vertex $\mathbf 1$ contains the point $\{g_{i_1},\dots, g_{i_n}\}\cup\bigl(\bigcup_{j>n} 1_{G_{i_j}}\bigr)\cup\bigl(\bigcup_{x\notin I} 1_{G_{x}}\bigr)$ which is at distance $2n$ of~$\mathbf 1$.
Therefore the action of $G\wr_XH$ on the connected component of~$\mathbf 1$ has an unbounded orbit and then $G\wr_XH$ does not have property FW.
\end{proof}
%
%
Once again, given a suitable $\mathbf{S}$, it is sometimes possible to give a direct proof of Lemma~\ref{Lemma:XFinite}.
For example, for Hilbert spaces one can take $\bigoplus_X\ell^2(G)$ with $\bigoplus_XG$ acting coordinatewise and $H$ by permutations.
On the other hand, both the forest structure on $\powerset{\bigsqcup_XG}$ and the ultrametric structure on $\bigoplus_XG$ are in general not invariant under the natural action of $H$ by permutations.

In fact, it follows from Theorems~\ref{Thm:FACK} and \ref{Thm:UncCoun} that in the assumptions of Lemma~\ref{Lemma:XFinite} it is not possible to replace property FW by property FA or by cof~$\neq\omega$.
%
%
\begin{rem}\label{Rem:Actionsb}
A reader familiar with wreath products might have recognized that we used the primitive action of the wreath product in the proof of Lemma~\ref{Lemma:XFinite}.

Indeed, $G$ acts on itself by left multiplication.
It hence acts on the set $G'\coloneqq G\sqcup\{\varepsilon\}$ by fixing $\varepsilon$, and we have the primitive action of $G\wr_{X} H$ on $G'^X$.
Now, the set $\bigsqcup_XG$ naturally embeds as the subset of $G'^X$ consisting of all functions $\varphi\colon X\to G'$ such that $\varphi(x)=\varepsilon$ for all but one $x\in X$.
This subset is $G\wr_{X} H$ invariant, which gives us the desired action of $G\wr_{X} H$ on $\bigsqcup_XG$.
\end{rem}
%
%
We now turn our attention to properties that behave well under cartesian products in the sense of Definition~\ref{Def:Cartesian}.

We first describe the comportement of property \BS{} under finite index subgroups.
%
%
\begin{lem}\label{Lemma:Subgroup}
Let $G$ be a group and let $H$ be a finite index subgroup.
\begin{enumerate}
\item
If $H$ has property \BS, then so has~$G$,
\item
If $\mathbf{S}$ has cartesian powers and $G$ has property \BS, then $H$ has property \BS.
\end{enumerate}
\end{lem}
\begin{proof}
Suppose that $G$ does not have \BS{} and let $X$ be a $\mathbf S$-space on which $G$ acts with an unbounded orbit $\orbite$.
Then $H$ acts on $X$ and $\orbite$ is a union of at most $[G:H]$ orbits. This directly implies that $H$ has an unbounded orbit and therefore does not have \BS.

On the other hand, suppose that $H\leq G$ is a finite index subgroup of G without property \BS.
Let $\alpha\colon H\curvearrowright X$ be an action of $H$ on a $\mathbf S$-space $(X,d_X)$ such that there is an unbounded orbit $\orbite$.
Similarly to the classical theory of representations of finite groups, we have the induced  action $\Ind_H^G(\alpha)\colon G \curvearrowright X^{G/H}$ on the set $X^{G/H}$. Since $H$ has finite index, $X^{G/H}$ is a $\mathbf S$-space and the action is by S-automorphisms. On the other hand, the subgroup $H\leq G$ acts diagonally on $X^{G/H}$, which implies that $\diag(\orbite)$ is contained in a $G$-orbit.
Since $\diag(\orbite)$ is unbounded, $G$ does not have property \BS.

For readers that are not familiar with representations of finite groups, here is the above argument in more details.
Let $(f_i)_{i=1}^n$ be a transversal for $G/H$.
The natural action of $G$ on $G/H$ gives rise to an action of $G$ on $\{1,\dots,n\}$.
Hence, for any $g$ in $G$ and $i$ in $\{1,\dots,n\}$ there exists a unique $h_{g,i}$ in $H$ such that $gf_i=f_{g.i}h_{g,i}$. That is, $h_{g,i}=f_{g.i}^{-1}gf_i$.
We then define $g.(x_1,\dots,x_n)\coloneqq(h_{g,g^{-1}.1}.x_{g^{-1}.1},\dots,h_{g,g^{-1}.n}.x_{g^{-1}.n})$. This is indeed an action by S-automorphisms on $X^{G/H}$ by Condition~\ref{Item:Product} of Definition~\ref{Def:Cartesian}.
Moreover, every element $h\in H$ acts diagonally by $h.(x_1,\dots,x_n)=(h.x_1,\dots,h.x_n)$.
In particular, this $G$ action has an unbounded orbit.
\end{proof}
%
%
We now prove one last lemma that will be necessary fo the proof of Theorem~\ref{Thm:Main}.
%
%
\begin{lem}\label{Lemma:Unboundedness}
Suppose that $\mathbf{S}$ has cartesian powers. If $X$ is finite and $G\wr_XH$ has property \BS, then $G$ has property \BS.
\end{lem}
\begin{proof}
Suppose that $G$ does not have \BS{} and let $(Y,d_Y)$ be a $\mathbf S$-space on which $G$ acts with an unbounded orbit $G.y$.
Then $(Y^X,d)$ is a $\mathbf S$-space and we have the \emph{primitive action} of the wreath product $G\wr_XH$ on $Y^X$:
\[
	\bigl((\varphi,h).\psi\bigr)(x)=\varphi(h^{-1}.x).\psi(h^{-1}.x).
\]
By Condition~\ref{Item:Product} of Definition~\ref{Def:Cartesian}, this action is by S-automorphisms.
The orbit $G.y$ embeds diagonally and hence $\diag(G.y)$ is an unbounded subset of some $G\wr_XH$-orbit, which implies that $G\wr_XH$ does not have property \BS.
\end{proof}
%
%
By combining Corollary~\ref{Cor:Wreath} and Lemmas~\ref{Lemma:XFinite} and~\ref{Lemma:Unboundedness} we obtain the following result which implies Theorem~\ref{Thm:Main}.
%
%
\begin{thm}\label{Thm:Technic}
Suppose that $\mathbf{S}$ has cartesian powers and is such that \BS{} implies FW.
Let $G$ and $H$ be two groups with $G$ non-trivial and let $X$ be a set on which $H$ acts. Then the wreath product $G \wr_X H$ has property \BS{} if and only if $G$ and $H$ have property \BS{} and $X$ is finite.
\end{thm}
%
%
We now proceed to prove Theorem~\ref{Thm:UncCoun}.
As for Lemma~\ref{Lemma:Cof}, it is also possible to prove it using the characterization of cof~$\neq\omega$ in terms of subgroups, but we will only give a proof using the characterization in terms of actions on ultrametric spaces.
%
%
\begin{thm}
Let $G$ and $H$ be two groups with $G$ non-trivial and let $X$ a set on which $H$ acts. Then the wreath product $G \wr_X H$ has cof~$\neq\omega$ if and only if $G$ and $H$ have cof~$\neq\omega$ and $H$ acts on $X$ with finitely many orbits.
\end{thm}
\begin{proof}
By Corollary~\ref{Cor:Wreath} and Lemma~\ref{Lemma:Cof} we already know that if $G \wr_X H$ has cof~$\neq\omega$, then $H$ has cof~$\neq\omega$ and it acts on $X$ with finitely many orbits.
We will now prove that if $G \wr_X H$ has cof~$\neq\omega$ so does $G$.
Let us suppose that $G$ has countable cofinality. By Lemma~\ref{Lemma:CofSub}, there exists an ultrametric $d$ on $G$ such that the action of $G$ on itself by left multiplication has an unbounded orbit.
But then we have the primitive action of the wreath product $G\wr_XH$ on $G^X\cong\prod_XG$, which preserves $\bigoplus_XG$.
It is easy to check that the map $d_\infty\colon\bigoplus_XG\times\bigoplus_XG\to\R$ defined by $d_\infty(\psi_1,\psi_2)\coloneqq\max\setst{d\bigl(\psi_1(x),\psi_2(x)\bigr)}{x\in X}$ is a $G\wr_XH$-invariant ultrametric.
Finally, let $h\in G$ be an element of unbounded $G$-orbit for $d$ and let $x_0$ be any element of $X$. Then for any $g$ in $G$ we have $(\delta_{x_0}^g,1).\delta_{x_0}^h=\delta_{x_0}^{gh}$ and hence $d_\infty(\delta_{x_0}^h,\delta_{x_0}^{gh})=d(h,gh)$ is unbounded.

Suppose now that both $G$ and $H$ have cof~$\neq\omega$ and that $H$ acts on $X$ with finitely many orbits. We want to prove that $G\wr_XH$ has cof~$\neq\omega$.

Let $(Y,d)$ be an ultrametric space on which $G\wr_XH$ acts.
Then $H$ and all the $G_x$ act on $Y$ with bounded orbits.
Let $\mathcal O_1,\dots,\mathcal O_n$ be the $H$-orbits on $X$ and for each $1\leq i\leq n$ choose an element $x_i$ in $\mathcal O_i$.
Let $y$ be any element of $Y$.
Then $H.y$ has finite diameter $D_0$ while $G_{x_i}.y$ has finite diameter $D_i$.
For any $x\in X$, there exists $1\leq i\leq n$ and $h\in H$ such that $x=h.x_i$.
We have
\begin{align*}
	d\bigl((\delta_{x_i}^g,h^{-1}).y,y\bigr)&\leq\max\{d\bigl((\delta_{x_i}^g,h^{-1}).y,(\delta_{x_i}^g,1).y\bigr),d\bigl((\delta_{x_i}^g,1).y,y\bigr)\}\\
	&=\max\{d\bigl((1,h^{-1}).y,y\bigr),d\bigl((\delta_{x_i}^g,1).y,y\bigr)\}\\
	&\leq \max\{D_0,D_i\},
\end{align*}
which implies that the diameter of $G_{x_i}h^{-1}.y$ is bounded by $\max\{D_0,D_i\}$.
But $G_{x_i}h^{-1}.y$ has the same diameter as $hG_{x_i}h^{-1}.y=G_{h.x_i}.y=G_x.y$.

On the other hand, the diameter of $\bigoplus_XG.y$ is bounded by the supremum of the diameters of the $G_{x_i}.y$, and hence bounded by $\max\{D_0,D_1,\dots,D_n\}$.
Finally, for $(\varphi,h)$ in $G\wr_YH$ we have
\begin{align*}
	d\bigl(y,(\varphi,h).y\bigr)&\leq\max\{d\bigl(y,(\varphi,1).y\bigr),d\bigl((\varphi,1).y,(\varphi,h).y\bigr)\}\\
	&=\max\{d\bigl(y,(\varphi,1).y\bigr),d\bigl(y,(1,h).y\bigr)\}\\
	&\leq\max\{\max\{D_0,D_1,\dots,D_n\},D_0\}.
\end{align*}
That is, the diameter of $G\wr_YH.z$ is itself bounded by $\max\{D_0,D_1,\dots,D_n\}$, which finishes the proof.
\end{proof}
%
%
While the fact that being a tree is not compatible with powers is an obstacle to our methods, we still have the following weak version of Theorem~\ref{Thm:Technic} for property FA.
%
%
\begin{prop}\label{Prop:WRFA}
Let $G$ and $H$ be two groups with $G$ non-trivial and $X$ a set on which $H$ acts.
Then
\begin{enumerate}
\item If $G\wr_XH$ has property FA, then $H$ has property FA, $H$ acts on $X$ with finitely many orbits, $G/[G:G]$ is finite and $G$ has cof~$\neq\omega$,
\item If both $G$ and $H$ have property FA and $X$ is finite, then $G\wr_XH$ has property~FA.
\end{enumerate}
\end{prop}
\begin{proof}
The only things that is not a consequence of Corollary~\ref{Cor:Wreath}, Lemma~\ref{Lemma:Cof} and Theorem~\ref{Thm:UncCoun} is the fact that if $G\wr_XH$ has property FA, then the abelianization $G^{\ab}=G/[G,G]$ is finite.

If $G\wr_XH$ has property FA, so does its abelianization $(G\wr_XH)^{\ab}=(G^{\ab})^{X/H}\times H^{\ab}$. Since we already know that $H$ acts on $X$ with finitely many orbits, we conclude that $G^{\ab}$ is an abelian group with property FA. % and hence finite.
The group $G^{\ab}$ is torsion, otherwise it would have a quotient isomorphic to $\Z$ which implies that $G^{\ab}$ does not have FA.
Since $G^{\ab}$ is abelian and torsion, by a result of Kaplansky we have $G\cong\bigoplus_{i\in I}H_i$ where each $H_i$ is a $p_i$-group for some prime $p_i$.
Either this direct sum is infinite, in which case $G^{\ab}$ does not have property FA, or $G^{\ab}$ has a quotient which is an infinite abelian $p$-group. But this is not possible as infinite abelian $p$-groups do not have cof~$\neq\omega$.
\end{proof}
%
%
Moreover, by using Lemma~\ref{Lemma:Cof} we can get ride of the ``finitely many orbits'' hypothesis in Theorem~\ref{Thm:FACK} in order to obtain Theorem~\ref{Thm:FAFiniteOrbits}.
%
%
%
%
%
%
%
%
%
%
\bibliography{biblio.bib}
\bibliographystyle{plain}
%
%
%
%
%
%
%
%
%
%
\clearpage
\paragraph{Groups acting with fixed point on \texorpdfstring{$\mathbf S$}{\textbf{S}}-spaces}\PH{J'ai mis cela ici, à voir si on en fait quelque chose.}
Some of the properties that are of interest for us have been historically defined via the existence of a fixed point for some action. More generally, we say that a group $G$ has \emph{property F$\mathbf S$} if any $G$ action on a $\mathbf S$-space has a fixed point.

Since our actions are by isometries, property F$\mathbf S$ implies property \BS. The other implication holds as soon as we have a suitable notion of the center of a (non-empty) bounded subset $X$.
This is for example the case for (real or complex) Hilbert spaces \cite[Lemma 2.2.7]{Bekka2008}. On the other hand, property FA is usually defined as ``any action without inversion on a tree as a fixed point'', which is equivalent to ``any action on a tree either fixes a point or fixes the middle of an edge''. It is easy exercise to prove that if an action of a group on a tree has one bounded orbit, then it fixes a point or the middle of an edge.
For action on (ultra)-metric spaces or on connected median graphs, F$\mathbf S$ is stricly stronger than \BS. Indeed, this trivially follows from the action by rotation of $C_4$ on the square graph.
However,  by \cite{MR1663779, Cornulier2013} if a group $G$ acts on a connected median graph with a bounded orbit, then it has a finite orbit.

\end{document}