%!TEX root = Property_FW_and_wreath_products.tex
\label{Section:Median}
In this section, we will investigate the property FW via the action of groups on median graphs.
%After recalling the definition of median graphs, we will prove some results 

For $u$ and $v$ two vertices of a connected graph $\mathcal G$, we define the total interval $[u,v]$ as the set of vertices that lies on some shortest path between $u$ and $v$.
A connected graph $\mathcal G$ is \emph{median} if for any three vertices $u$, $v$, $w$, the intersection $[u,v]\cap[v,w]\cap[u,w]$ consists of a unique vertex, denoted $m(u,v,w)$.
A graph is median if each of its connected components is median.

Recall that a group $G$ has property FW if and only if every $G$ action on a connected median graph has bounded orbits.

We now prove a series of results that will be generalized to a broader context in Section~\ref{Section:Generalizations}.
The following easy result is folklore, and we provide a proof only for the sake of completeness.
\begin{lem}\label{Lemma:Subgroup}
Let $G$ be a group and $H$ be a finite index subgroup.
If $H$ has FW, then so does~$G$.
\end{lem}
\begin{proof}
Suppose that $G$ does not have FW and let $X$ be a connected median graph on which $G$ acts with unbounded orbits.
Then $H$ acts on $X$ and every $G$-orbit is a union of at most $[G:H]$ orbits. This directly implies that $H$ acts on $X$ with unbounded orbit and that $H$ does not have FW.
\end{proof}

We also have the following lemma on semi-direct products:
\begin{lem}\label{Lemma:Semidirect}
Let $G=N\rtimes H$ be a semidirect product. Then
\begin{enumerate}
\item
If $G$ has FW, then so does $H$.
\item
If both $N$ and $H$ have FW, then $G$ also has FW.
\end{enumerate}
\end{lem}
\begin{proof}
Suppose that $G$ has FW and let $X$ be a non-empty connected median graph on which $H$ acts.
Then $G$ acts on $X$ by $g.x\coloneqq h.x$ where $g=nh$ with $n\in N$ and $h\in H$.
By assumption, the action of $G$ on $X$ has bounded orbits and so does the action of $H$.

On the other hand, suppose that $N$ and $H$ have FW and let $X$ be a non-empty connected median graph on which $G$ acts.
Then both $N$ and $H$ acts on $X$, with orbits bounded respectively by $d_N$ and $d_H$.
Now, for every $x\in X$ and $g\in G$, there is $n\in N$ and $h\in H$ such that $g=nh$ ans thus $g.x=n.(h.x)$ is at distance at most $d_N+d_H$ from $x$.
\end{proof}
Multiple applications of the above Lemma give us
\begin{cor}\label{Cor:Wreath}
Let $G\wr_X H$ be the wreath product of $G$ and $H\curvearrowright X$.
Then
\begin{enumerate}
\item
If $G\wr_X H$ has FW, then so does $H$.
\item
If $G$ and $H$ have FW and $H$ acts on a finite set $X$, then $G\wr_X H$ has FW.
\end{enumerate}
\end{cor}







We now turn back our attention on results that will rely more on the fact that we are using median graphs.

Trees are the simplest examples of median graphs and a simple verification shows that if $\mathcal G$ and $\mathcal H$ are both median graphs, then their cartesian product is also median.
On the other hand, the following example will be fundamental for us.
\begin{exmp}\label{Ex:MainMedian}
Let $X$ be a set and let  $\mathcal P_f(X)$ be the set of all its finite subsets.
Define a graph structure on $\mathcal P_f(X)$ by putting an edge between $E$ and $F$ if and only if $\#(E\Delta F)=1$, where $\Delta$ is the symmetric difference.
Therefore, the distance between two subsets $E$ and $F$ is $E\Delta F$, while $[E,F]$ consist of all subsets of $X$ that both contain $E\cap F$ and are contained in $E\cup F$.
In particular $\mathcal P_f(X)$ is a connected median graph, with $m(D,E;F)$ being the set of all elements belonging to at least two of $D$, $E$ and $F$. In other words,
$m(D,E;F)=(D\cap E)\cup(D\cap F)\cup(E\cap F)$.

The same construction endows $\mathcal P(X)$ the set of all subsets of $X$ with a structure of median graph, where the connected component of $E$ consists of all $F$ such that $E\Delta F$ is finite.
\end{exmp}

The graphs $\mathcal P_f(X)$ are exactly the hypercube and it turns out that every connected median graph is a retract of some $\mathcal P_f(X)$, see \cite{Bandelt1984}.

These graphs will be fundamental for us due to the following fact.
Any action of a group $G$ on a set $X$ naturally extends to an action of $G$ on $\mathcal P_f(X)$ by graph homomorphisms: $g.\{x_1,\dots,x_n\}=\{g.x_1,\dots,g.x_n\}$.

Building on Example~\ref{Ex:MainMedian}, we obtain that no infinite sum of groups has the property FW.
\begin{lem}\label{Lemma:Sum}
An infinite sum of non-trivial groups $G=\bigoplus_{i=1}^\infty G_i$ does not have FW.
\end{lem}
\begin{proof}
Let $X=\bigsqcup_{i=1}^\infty G_i$.
There is a natural action of $G$ on $X$: $G_i$ acts by left multiplication on $G_i$ and trivially on $G_j$ for $j\neq i$.
Therefore, we have an action of $G$ on the connected median graph $\mathcal P_f(X)$.
For every $i$, choose a non-trivial $g_i\in G_i$.
Then the orbit of the vertex $\{1_{G_1},\dots, 1_{G_n}\}$ contains the point $\{g_1,\dots, g_n\}$ which is at distance $2n$ of $\{1_{G_1},\dots, 1_{G_n}\}$.
That is the action of $G$ on $\mathcal P_f(X)$ has unbounded orbits.
\end{proof}

As a direct corollary of Lemmas \ref{Lemma:Sum} and \ref{Lemma:Semidirect}, we obtain
\begin{cor}\label{Cor:Sum}
The group $\otimes_X G$ has FW if and only if $X$ is finite and $G$ has FW.
\end{cor}


We also have a converse of Lemma~\ref{Lemma:Subgroup}.
\begin{lem}\label{Lemma:Subgroup2}
Let $G$ be a group and $H$ be a finite index subgroup.
If $G$ has FW, then so does~$H$.
\end{lem}
\begin{proof}
Suppose that $H$ does not have FW and let $X$ be a connected median space on which $H$ acts with unbounded orbits.
This induces an action of $G$ on $X^{G/H}$ by\todo{Finish the proof}
\end{proof}







We finally characterize which wreath products do have FW and hence provide a proof of Theorem~\ref{Thm:Main}.
\begin{prop}\label{Prop:Median}
Let $G,H$ be two discrete groups with $G$ non-trivial and $X$ a set on which $H$ acts. The wreath product $G \wr_X H$ has the property FW if and only if $G$ and $H$ have the property FW and $X$ is finite.
\end{prop}
\begin{proof}
In view of Corollary~\ref{Cor:Wreath} it remains to show that if $G\wr_X H$ has FW, then $G$ has FW and $X$ is finite.

First, suppose that $G$ does not have FW and let $Y$ be a connected median graph on which $G$ acts with unbounded orbits.
Then $G\wr_X H$ acts on the connected median graph $Y\times \mathcal P_f(X)$ by
\[
	\bigl(\phi,h\bigr).(y,E)=
	\begin{cases}
	(\phi(h.t).y,h.\{t\})&\textnormal{if $E=\{t\}$}\\
	(y,h.E)&\textnormal{if $E$ is not a singleton}
	\end{cases}
\]
But then the orbit of $(y,\{x\})$ is unbounded for every $x\in X$, which implies that $G\wr_X H$ does not have FW.
Indeed, it contains
\[
	\{(\delta_{x}^g,1).(y,\{x\})=(g.y,\{x\})\,|\,g\in G\},
\]
where $\delta_{x}^g(x)=g$ and $\delta_{x}^g(z)=1$ if $z\neq x$. 
And since all the $\{g.y,\{x\}\}_{g\in G}$ are in the same slice, the distance between them is the same as the distance between the $\{g.y\}_{g\in G}$.

Finally, suppose that $X$ is infinite and let $g$ be a non-trivial element of $G$.
Then $G$ acts naturally on the connected median graph $P_f(\bigsqcup_XG)$ and therefore $G\wr_X H$ acts on $\mathcal P_f(\bigsqcup_XG)\times \mathcal P_f(X)$.
Let $v$ be the the vertex $(\{1_1,\dots, 1_n\},\emptyset)$, where $\{1_1,\dots, 1_n\}$ consists of $n$ times the element $1_G\in G$ but living in distinct copies of $G$.
The orbit of $v$ contains the point $(\{g_1,\dots, g_n\},\emptyset)$ which is at distance $2n$ of $v$.
That is the action of $G\wr_X H$ on $\mathcal P_f(\bigsqcup_XG)\times \mathcal P_f(X)$ has unbounded orbits and $G\wr_X H$ has not FW.
\end{proof}






\section{Related concepts and generalizations}\label{Section:Generalizations}
This short section is devoted to generalize some of the results of Section~\ref{Section:Median} as well as to presente some properties related to the FW property.

Property FW does not stand alone and there are other similar properties that are of great interest.
\begin{defn}\label{Def:FHFA}
Let $G$ be a group.
It is said to have \emph{property SB} if any action on a metric space has bounded orbits.
It has \emph{property FH} if any action on a real Hilbert space has bounded orbits and \emph{property FA} if any action on a tree has bounded orbits.
Finally, $G$ is said to have (cofinality$\neq\omega$) if any action on a ultrametric space has bounded orbits.
\end{defn}
For countable groups (and more generally for $\sigma$-compact locally compact groups), property FH is equivalent to the celebrated Kazdhan's property (T) by the Delorme-Guichardet theorem, see for example \cite{MR2415834}, but this is not true in general \cite{MR2240370}.
\todo[inline]{Maybe expand a little more, or at least give some references}
The names FH, FW and FA come from the fact that this property admit a description in terms of existence of a Fixed point for action on Hilbert spaces, on spaces with Walls and on trees (\emph{Arbres} in french).
On the other hand, SB stands for Strongly Bounded and is sometimes called the Bergman property, while a group has cofinality$\neq\omega$ if and only if it cannot be written has an increasing union of proper subgroups.

We have the following strict implications \cite{MR1432323,MR0476875 ,MR3299841,2013arXiv1302.5982C}
\[
\textnormal{SB}\implies \textnormal{FH}\implies \textnormal{FW}\implies \textnormal{FA}\implies(\textnormal{cofinality}\neq\omega).
\]

It is possible to define other properties in the spirit of Definition~\ref{Def:FHFA}.
Let P be a property of metric spaces (for example \emph{be a connected median graph}) and BP be the group property: \emph{Every $G$-action on a space with P has bounded orbits}, where a $G$-action is supposed to ``preserve the P-structure''.
Lemmas \ref{Lemma:Subgroup} and \ref{Lemma:Semidirect} as well as Corollary \ref{Cor:Wreath} as well as their proofs remain true for groups with property BP.

On the other hand, the other results of Section~\ref{} require a specific construction and do not generalize straightforward to groups with property BP.
Nevertheless, it is possible to extract the main ingredients of the proof and to adapt them in some specific cases.
We will now give a raw outline of this process, but let the details to the interested reader.

\todo[inline]{Réécrire correctement ce qui suit. Veux-t-on parler de foncteurs ?}
In order to generalize Example~\ref{Ex:MainMedian}, we will need to construct from a $G$ action on a set $X$ a $G$ action on a P-space $Y$.
More precisely, we say that it is \emph{possible to extend $G$ actions to P-spaces} if there is a $G$-equivariant map $i$ that associate to every $G$-set $X$ a P-space $Y$ endowed with a $G$ action and a $G$-equivariant map $\iota\colon X\hookrightarrow Y$.
For connected median graphs we took $Y=\mathcal P_f(X)$ in Example~\ref{Ex:MainMedian}, while for real Hilbert spaces it is possible to take $Y$ to be the real Hilbert space generated by $X$. That is, 
\[Y=\ell^2(X)=\bigg\{f\colon X\to \mathbf R\,\bigg|\, \sum_{x\in X}f(x)^2<\infty\bigg\}.\]
For property SB, we can take $Y$ to be  $\mathcal P_f(X)$ or $\ell^2(X)$, as well as many other possibilities.
In the context of trees, is is possible to take $Y=X\sqcup\{*\}$ with for every $x\in X$ an edge between $x$ and $*$, while fo ultrametric spaces it is always possible to take $Y=X$ with the discrete metric.

Let P be a property such that $G$ actions extend to P-spaces.
%We would like to generalize Lemma \ref{Lemma:Sum} and Corollary \ref{Cor:Sum} to groups with property BP.
Observe that the map $\iota\colon X\hookrightarrow Y$ naturally extends to $\iota_*\colon \mathcal P_f(X)\hookrightarrow \mathcal P_f(Y)$, where $\iota_*(\{x_1,\dots,x_n\})=\{\iota(x_1),\dots,\iota(x_n)\}$.
Define
\[d_n\coloneqq\inf\{d(\iota_*(\{x_1,\dots,x_n\}),\iota_*(\{y_1,\dots,y_n))\,|\,\{x_1,\dots,x_n\}\cap\{y_1,\dots,y_n\}=\emptyset \}\]
\todo[inline]{Regarder comment écrire correctement la condition pour que le lemme \ref{Lemma:Sum} fonctionne.}
We will say that P is a property such that $G$ actions extend \emph{unboundedly} to P-spaces if

\todo[inline]{Vérifier ce qui suit. Écrire le thm pour les SB spaces et regarder pour des références (Cornulier)}
On the other hand, Lemma~\ref{Lemma:Subgroup2} only need that a product of two P-spaces is still a P-space, which holds for real Hilbert spaces, but once again fails for trees.

Finally, Proposition~\ref{Prop:Median} needs both that a product of two P-spaces is still a P-space and that is is possible to extend $G$ actions to P-spaces.
In particular, we have
\begin{prop}
Let $G,H$ be discret groups and $X$ a $H$-set. Then the wreath product $G \wr_X H$ has the property (FH) if and only if $G$ and $H$ have the property (FH) and if $X$ is finite
\end{prop}
\todo[inline]{Regarder si on peut enlever groupe discret et ne pas passer par le résultat pour (T)}
The proof is a direct application of the equivalence of the properties (FH) and (T) for discrete groups (see \cite{Bekka2008}) and of the Theorem \ref{T:Wreath_prop_T} .
%
\begin{thm}[\cite{Cornulier2011}]
Let $G,H$ be two groups and $X$ be a non-empty $H$ set which has only a finite number of orbits and no fixed point. Then the wreath product $G \wr_X H$ has the property (FA) if and only if $H$ has the property (FA) and if $G$ is a group with a finite abelianization, which cannot be expressed as an union of proper increasing sequence of subgroups.
\end{thm}
\todo[inline]{regarder si c'est plus fort que $H$ possède la propriété (FA) et le lien avec le cardinal de $X$}

\todo[inline]{Regarder le cas PW, Haagerup,... i.e. action propre. Rappel: une action isométrique de $G$ est propre si pour tout $x$ (de manière équivalente il existe $x$), pour tout $r\in \mathbf R$ l'ensemble $\{g\in G\,|\,d(x,g.x)\}$ est fini.}
