%!TEX root = Property_FW_and_wreath_products.tex
In this section, we will investigate the property FW via the action of groups on median graphs.

For $u$ and $v$ two vertices of a connected graph $\mathcal G$, we define the total interval $[u,v]$ as the set of vertices that lies on some shortest path between $u$ and $v$.
A connected graph $\mathcal G$ is \emph{median} if for any three vertices $u$, $v$, $w$, the intersection $[u,v]\cap[v,w]\cap[u,w]$ consists of a unique vertex, denoted $m(u,v,w)$.
A graph is median if each of its connected components is median.

A simple verification shows that if $\mathcal G$ and $\mathcal H$ are both median graphs, then their cartesian product is also median.

\todo[inline]{Réfléchir à comment écrire la section. 2 possibilités. 1) Grouper par type de résultat (semi-direct, etc.) ou 2) grouper par ce que se généralise à la propriété P et ce qui est propre au graphe médian.}
The following example will be fundamental for us.
\begin{exmp}
Let $X$ be a set and let  $\mathcal P_f(X)$ be the set of all its finite subsets.
Define a graph structure on $\mathcal P_f(X)$ by putting an edge between $E$ and $F$ if and only if $\#(E\Delta F)=1$, where $\Delta$ is the symmetric difference.
Therefore, the distance between two subsets $E$ and $F$ is $E\Delta F$, while $[E,F]$ consist of all subsets of $X$ that both contain $E\cap F$ and are contained in $E\cup F$.
In particular $\mathcal P_f(X)$ is a connected median graph, with $m(D,E;F)$ being the set of all elements belonging to at least two of $D$, $E$ and $F$. In other words,
$m(D,E;F)=(D\cap E)\cup(D\cap F)\cup(E\cap F)$.

The same construction endows $\mathcal P(X)$ the set of all subsets of $X$ with a structure of median graph, where the connected component of $E$ consists of all $F$ such that $E\Delta F$ is finite.
\end{exmp}

The graphs $\mathcal P_f(X)$ are exactly the hypercube and it turns out \cite{}\todo{Put refs} that every connected median graph is a retract of some $\mathcal P_f(X)$.

These graphs will be fundamental for us due to the following fact.
Any action of a group $G$ on a set $X$ naturally extends to an action of $G$ on $\mathcal P_f(X)$ by graph homomorphisms: $g.\{x_1,\dots,x_n\}=\{g.x_1,\dots,g.x_n\}$.


Building on this, we obtain that no infinite sum of groups has the property FW.
\begin{lem}\label{Lemma:Sum}
An infinite sum of non-trivial groups $G=\bigoplus_{i=1}^\infty G_i$ does not have FW.
\end{lem}
\begin{proof}
Let $X=\bigsqcup_{i=1}^\infty G_i$.
There is a natural action of $G$ on $X$: $G_i$ acts by left multiplication on $G_i$ and trivially on $G_j$ for $j\neq i$.
Therefore, we have an action of $G$ on the connected median graph $\mathcal P_f(X)$.
For every $i$, choose a non-trivial $g_i\in G_i$.
Then the orbit of the vertex $\{1_{G_1},\dots, 1_{G_n}\}$ contains the point $\{g_1,\dots, g_n\}$ which is at distance $2n$ of $\{1_{G_1},\dots, 1_{G_n}\}$.
That is the action of $G$ on $\mathcal P_f(X)$ has unbounded orbits.
\end{proof}

\begin{lem}\label{Lemma:Subgroup}\todo{Mettre une phrase d'intro; ce n'est pas nouveau}
Let $G$ be a group and $H$ be a finite index subgroup.
If $H$ has FW, then so does~$G$.
\end{lem}
\begin{proof}
Suppose that $G$ does not have FW and let $X$ be a connected median graph on which $G$ acts with unbounded orbits.
Then $H$ acts on $X$ and every $G$-orbit is a union of at most $[G:H]$ orbit. This directly implies that $H$ acts on $X$ with unbounded orbit and that $H$ does not have FW.
\end{proof}
\begin{lem}
Let $G$ be a group and $H$ be a finite index subgroup.
If $G$ has FW, then so does~$H$.
\end{lem}
\begin{proof}
Suppose that $H$ does not have FW and let $X$ be a connected median space on which $H$ acts with unbounded orbits.
Then $G$ acts on $X^{G/H}$ by 
\end{proof}

We also have the following easy result on semi-direct products:
\begin{lem}\label{Lemma:Semidirect}
Let $G=N\rtimes H$ be a semidirect product. Then
\begin{enumerate}
\item
If $G$ has FW, then so does $H$.
\item
If both $N$ and $H$ have FW, then $G$ also has FW.
\end{enumerate}
\end{lem}
\begin{proof}
Suppose that $G$ has FW and let $X$ be a non-empty connected median graph on which $H$ acts.
Then $G$ acts on $X$ by $g.x\coloneqq h.x$ where $g=nh$ with $n\in N$ and $h\in H$.
By assumption, the action of $G$ on $X$ has bounded orbits and so does the action of $H$.

On the other hand, suppose that $G$ does not have FW and let $X$ be a non-empty connected median graph on which $G$ acts.
Then both $N$ and $H$ acts on $X$, with orbits bounded respectively by $d_N$ and $d_H$.
Now, for every $x\in X$ and $g\in G$, there is $n\in N$ and $h\in H$ such that $g=nh$ ans thus $g.x=n.(h.x)$ is at distance at most $d_N+d_H$ from $x$.
\end{proof}
Multiple applications of the above Lemma give us
\begin{cor}\label{Cor:Wreath}
Let $G\wr_X H$ be the wreath product of $G$ and $H\curvearrowright X$.
Then
\begin{enumerate}
\item
If $G\wr_X H$ has FW, then so does $H$.
\item
If $G$ and $H$ have FW and $H$ acts on a finite set $X$, then $G\wr_X H$ has FW.
\end{enumerate}
\end{cor}


It also follows from Lemmas \ref{Lemma:Sum} and \ref{Lemma:Semidirect} that
\begin{cor}
The group $\otimes_X G$ has FW if and only if $X$ is finite and $G$ has FW.
\end{cor}

We finally characterize which wreath products do have FW.
\begin{prop}\label{Prop:Median}
Let $G$ be a non-trivial group.
Then the group $G\wr_X H$ has FW if and only if both $G$ and $H$ have FW and $X$ is finite.
\end{prop}
\begin{proof}
In view of Corollary~\ref{Cor:Wreath} it remains to show that if $G\wr_X H$ has FW, then $G$ has FW and $X$ is finite.

First, suppose that $G$ does not have FW and let $Y$ be a connected median graph on which $G$ acts with unbounded orbits.
Then $G\wr_X H$ acts on the connected median graph $Y\times \mathcal P_f(X)$ by
\[
	\bigl(\phi,h\bigr).(y,E)=
	\begin{cases}
	(\phi(h.t).y,h.\{t\})&\textnormal{if $E=\{t\}$}\\
	(y,h.E)&\textnormal{if $E$ is not a singleton}
	\end{cases}
\]
But then the orbit of $(y,\{x\})$ is unbounded for every $x\in X$, which implies that $G\wr_X H$ does not have FW.
Indeed, it contains
\[
	\{(\delta_{x}^g,1).(y,\{x\})=(g.y,\{x\})\,|\,g\in G\},
\]
where $\delta_{x}^g(x)=g$ and $\delta_{x}^g(z)=1$ if $z\neq x$. 
And since all the $\{g.y,\{x\}\}_{g\in G}$ are in the same slice, the distance between them in the same as the distance between the $\{g.y\}_{g\in G}$.

Finally, suppose that $X$ is infinite and let $g$ be a non-trivial element of $G$.
Then $G$ acts naturally on the connected median graph $P_f(\bigsqcup_XG)$ and therefore $G\wr_X H$ acts on $\mathcal P_f(\bigsqcup_XG)\times \mathcal P_f(X)$.
Let $v$ be the the vertex $(\{1_1,\dots, 1_n\},\emptyset)$, where $\{1_1,\dots, 1_n\}$ consists of $n$ times the element $1_G\in G$ but living in distinct copies of $G$.
The orbit of $v$ contains the point $(\{g_1,\dots, g_n\},\emptyset)$ which is at distance $2n$ of $v$.
That is the action of $G\wr_X H$ on $\mathcal P_f(\bigsqcup_XG)\times \mathcal P_f(X)$ has unbounded orbits and $G\wr_X H$ has not FW.
\end{proof}

We conclude this section by the following important remark that some of the above results apply in a more general context.
\begin{rem}
Let Q be a property of metric spaces (for example \emph{be a connected median graph}) and P be the group property: \emph{Every $G$-action on a space with Q has bounded orbits.}
Then Lemmas \ref{Lemma:Subgroup} and \ref{Lemma:Semidirect} as well as Corollary \ref{Cor:Wreath} are true for groups with P.

Example of such properties P are: property FW, property FA (every action on a tree has bounded orbits) or property FH (every isometric action on a real Hilbert space has bounded orbits) which for countable group is equivalent to Kazdhan's property (T).
\end{rem}
\todo[inline]{Dire qqch (même juste références) sur l'équivalent de la proposition \ref{Prop:Median} pour FA et FH}

\todo[inline]{Regarder le cas PW, Haagerup,... i.e. action propre. Rappel: une action isométrique de $G$ est propre si pour tout $x$ (de manière équivalente il existe $x$), pour tout $r\in \mathbf R$ l'ensemble $\{g\in G\,|\,d(x,g.x)\}$ est fini.}
