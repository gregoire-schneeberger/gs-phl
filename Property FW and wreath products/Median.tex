%!TEX root = Property_FW_and_wreath_products.tex
\label{Section:Median}
In this section, we will investigate the property FW via the action of groups on median graphs.
%After recalling the definition of median graphs, we will prove some results 

For $u$ and $v$ two vertices of a connected graph $\mathcal G$, we define the total interval $[u,v]$ as the set of vertices that lies on some shortest path between $u$ and $v$.
A connected graph $\mathcal G$ is \emph{median} if for any three vertices $u$, $v$, $w$, the intersection $[u,v]\cap[v,w]\cap[u,w]$ consists of a unique vertex, denoted $m(u,v,w)$.
%A graph is median if each of its connected components is median.

Recall that a group $G$ has property FW if and only if every $G$ action on a connected median graph has bounded orbits.\todo{This is not equivalent to have uniformly bounded orbits. Need to check everything.}

We now prove a series of results that will be generalized to a broader context in Section~\ref{Section:Generalizations}.
The following easy result is folklore, and we provide a proof only for the sake of completeness.
\begin{lem}\label{Lemma:Subgroup}
Let $G$ be a group and $H$ be a finite index subgroup.
If $H$ has FW, then so does~$G$.
\end{lem}
\begin{proof}
Suppose that $G$ does not have FW and let $X$ be a connected median graph on which $G$ acts with an unbounded orbit $\orbite$.
Then $H$ acts on $X$ and $\orbite$ is a union of at most $[G:H]$ orbits. This directly implies that $H$ has an unbounded orbit and therefore does not have FW.
\end{proof}

We also have the following lemma on semi-direct products:
\begin{lem}\label{Lemma:Semidirect}
Let $G=N\rtimes H$ be a semidirect product. Then
\begin{enumerate}
\item
If $G$ has FW, then so does $H$.
\item
If both $N$ and $H$ have FW, then $G$ also has FW.
\end{enumerate}
\end{lem}
\begin{proof}
Suppose that $G$ has FW and let $X$ be a non-empty connected median graph on which $H$ acts.
Then $G$ acts on $X$ by $g.x\coloneqq h.x$ where $g=nh$ with $n\in N$ and $h\in H$.
By assumption, the action of $G$ on $X$ has bounded orbits and so does the action of $H$.

On the other hand, suppose that $N$ and $H$ have FW and let $X$ be a non-empty connected median graph on which $G$ acts.
Then both $N$ and $H$ acts on $X$ with bounded orbits.
Let $x$ be an element of $X$, $d_1$ the diameter of $H.x$ and $d_2$ the diameter of $N.x$.
Since $G$ acts by isometries, for every $h$ in $H$ the set $N.(h.x)=Nh.x=hN.x=h.(N.x)$ has also diameter $d_2$.
Therefore, every element of $G.x=NH.x$ is at distance at most $d_1+d_2$ of $x$, which implies that the orbit $G.x$ is bounded with diameter at most $2(d_1+d_2)$.
\end{proof}
Multiple applications of the above Lemma give us
\begin{cor}\label{Cor:Wreath}
Let $G\wr_X H$ be the wreath product of $G$ and $H\curvearrowright X$.
Then
\begin{enumerate}
\item
If $G\wr_X H$ has FW, then so does $H$.
\item
If $X$ is finite and both $G$ and $H$ have FW, then $G\wr_X H$ has FW.
\end{enumerate}
\end{cor}







We now turn back our attention on results that will rely more on the median structure.

Trees are the simplest examples of median graphs and a simple verification shows that if $\mathcal G$ and $\mathcal H$ are both (connected) median graphs, then their cartesian product is also a (connected) median graph.
On the other hand, the following example will be fundamental for us.
%
%
\begin{exmp}\label{Ex:MainMedian}
Let $X$ be a set and let  $\powerset{X}=2^X$ be the set of all subsets of $X$.
Define a graph structure on $\powerset{X}$ by putting an edge between $E$ and $F$ if and only if $\#(E\Delta F)=1$, where $\Delta$ is the symmetric difference.
Therefore, the distance between two subsets $E$ and $F$ is $E\Delta F$ and
the connected component of $E$ is the set of all subsets $F$ with $E\Delta F$ finite.
For $E$ and $F$ in the same connected component, $[E,F]$ consist of all subsets of $X$ that both contain $E\cap F$ and are contained in $E\cup F$.
In particular, $\powerset{X}$ is a median graph, with $m(D,E,F)$ being the set of all elements belonging to at least two of $D$, $E$ and $F$. In other words, $m(D,E,F)=(D\cap E)\cup(D\cap F)\cup(E\cap F)$.
\end{exmp}
%
%
We denote by $\powersetf{X}$, respectively $\powersetcof{X}$ the set of all finite, respectively cofinite, subsets of $X$.
They are connected components of $\powerset{X}$, which coincide if and only if $X$ is finite

The connected components of $\powerset{X}$ are exactly the hypercubes and it turns out that every connected median graph is a retract of some $\powersetf{X}$, see \cite{Bandelt1984}.

These graphs will be fundamental for us due to the following fact.
Any action of a group $G$ on a set $X$ naturally extends to an action of $G$ on $\powerset{X}$ by graph homomorphisms: $g.\{x_1,\dots,x_n\}=\{g.x_1,\dots,g.x_n\}$.
Be careful that the action of $G$ on $\powerset{X}$ may exchange the connected components.
In fact, the connected component of $E\subset X$ is stabilized by $G$ if and only if $E$ is \emph{commensurated} by $G$, that is if for every $g\in G$ the set $E\Delta gE$ is finite.
For example, $\powersetf{X}$ and $\powersetcof{X}$ are always preserved by the $G$-action.

Building on Example~\ref{Ex:MainMedian}, we obtain that no infinite sum of groups has the property FW.
\begin{lem}\label{Lemma:Sum}
An infinite direct sum of non-trivial groups does not have FW.
\end{lem}
\begin{proof}
Let $I$ be an infinite set of indices, $(G_i)_{i\in I}$ be non-trivial groups and $G=\bigoplus_{i\in I} G_i$.
Let $X\coloneqq\bigsqcup_{i\in I} G_i$.
There is a natural action of $G$ on $X$: $G_i$ acts by left multiplication on $G_i$ and trivially on $G_j$ for $j\neq i$.
Therefore, we have an action of $G$ on the median graph $\powerset{X}$.
Let $\mathbf 1\coloneqq\bigcup 1_{G_i}$ be the subset of $X$ consisting of the identity elements of all the $G_i$.
Since every element of $G$ has only a finite number of non-trivial coordinates, the action of $G$ preserves the connected components of $\mathbf 1$ (and in fact every connected component of $\powerset{X}$).

Since $I$ is infinite, it contains an infinite countable subset $I_c=\{i_1,i_2,\dots\}$.
For every $j\in \N$, choose a non-trivial $g_j\in G_{i_j}$.
Then the orbit of the vertex $\mathbf 1$ contains the point $\{g_1,\dots, g_n\}\cup\bigcup_{i\notin\{i_1,\dots,i_n\}} 1_{G_{i}}$ which is at distance $2n$ of $\mathbf 1$.
That is the action of $G$ on $\powerset{X}$ has an unbounded orbit.
\end{proof}

As a direct corollary of Lemmas \ref{Lemma:Sum} and \ref{Lemma:Semidirect}, we obtain
\begin{cor}\label{Cor:Sum}
The group $\otimes_X G$ has FW if and only if $X$ is finite and $G$ has FW.
\end{cor}


We also have a converse of Lemma~\ref{Lemma:Subgroup}.
\begin{lem}\label{Lemma:Subgroup2}
Let $G$ be a group and $H$ be a finite index subgroup.
If $G$ has FW, then so does~$H$.
\end{lem}
\begin{proof}
Suppose that $H$ does not have FW and let $\alpha\colon H\curvearrowright X$ be an action of $H$ on a connected median space such that there is an unbounded orbit $\orbite$.
Similarly to the classical theory of representation of finite groups, we have the induced  action $\Ind_H^G(\alpha)\colon G \curvearrowright X^{G/H}$ on the graph $X^{G/H}$. Since $H$ has finite index, the graph $X^{G/H}$ is median. On the other hand, the subgroup $H\leq G$ acts diagonaly on $X^{G/H}$, which gives us a $H$-equivariant isometric embedding from $H\curvearrowright X$ to $H\curvearrowright X^{G/H}$ sending $\orbite$ to an unbounded $H$-orbit on $X^{G/H}$, which implies that $G$ does not have FW.

For readers that are not familiar with representations of finite groups, here is the above argument in more details.
Let $(f_i)_{i=1}^n$ be a transversal for $G/H$.
The natural action of $G$ on $G/H$ gives rise to an action of $G$ on $\{1,\dots,n\}$.
Hence, for any $g$ in $G$ and $i$ in $\{1,\dots,n\}$ there exists a unique $h_{g,i}$ in $H$ such that $gf_i=f_{g.i}h_{g,i}$. That is, $h_{g,i}=gf_if_{g.i}^{-1}$.
We then define $g.(x_1,\dots,x_n)\coloneqq(h_{g,g^{-1}.1}.x_{g^{-1}.1},\dots,h_{g,g^{-1}.n}.x_{g^{-1}.n})$. This is indeed an action, which preserves the graph structure of $X^{G/H}$ and such that every element $h\in H$ acts diagonally by $h.(x_1,\dots,x_n)=(h.x_1,\dots,h.x_n)$.
In particular, this $G$ action has an unbounded orbit.
\end{proof}







We finally characterize which wreath products do have FW, and hence provide a proof of Theorem~\ref{Thm:Main}.
\begin{prop}\label{Prop:Median}
Let $G$, $H$ be two discrete groups with $G$ non-trivial and $X$ a set on which $H$ acts. The wreath product $G \wr_X H$ has the property FW if and only if $G$ and $H$ have the property FW and $X$ is finite.
\end{prop}
\begin{proof}
In view of Corollary~\ref{Cor:Wreath} it remains to show that if $G\wr_X H$ has FW, then $G$ has FW and $X$ is finite.

First, suppose that $G$ does not have FW and let $Y$ be a connected median graph on which $G$ acts with an unbounded orbit $G.y$.
Then $G\wr_X H$ acts on the connected median graph $Y\times \powersetf{X}$ by
\[
	\bigl(\phi,h\bigr).(y,E)=
	\begin{cases}
	(\phi(h.t).y,h.\{t\})&\textnormal{if $E=\{t\}$}\\
	(y,h.E)&\textnormal{if $E$ is not a singleton.}
	\end{cases}
\]
But then the orbit of $(y,\{x\})$ is unbounded for every $x\in X$, which implies that $G\wr_X H$ does not have FW.
Indeed, this orbit contains
\[
	\{(\delta_{x}^g,1).(y,\{x\})=(g.y,\{x\})\,|\,g\in G\},
\]
where $\delta_{x}^g(x)=g$ and $\delta_{x}^g(z)=1$ if $z\neq x$.
Since $\{g.y\}_{g\in G}$ is unbounded, so is $\{g.y,\{x\}\}_{g\in G}$.

Finally, suppose that $X$ is infinite.
As in Lemma~\ref{Lemma:Sum}, the group $\bigoplus_XG$ acts coordinatewise on  $\bigsqcup_XG$. On the other hand, $H$ acts on $\bigsqcup_XG$ by permutation of the factors.
Altogether we have an action of $G\wr_XH$ on $\bigsqcup_XG$ and hence on the median graph $\powerset{\bigsqcup_XG}$.
Let $\mathbf 1\coloneqq\bigcup_{x\in X} 1_{G}$ be the subset of $\powerset{\bigsqcup_XG}$ consisting of the identity elements of all the copies of $G$.
As in Lemma~\ref{Lemma:Sum} we have that the action of $G\wr_XH$ on $\powerset{\bigsqcup_XG}$ preserves the connected component of $\mathbf 1$
and that the orbit of $\mathbf 1$ is unbounded.
\end{proof}

\begin{rem}
A reader familiar with wreath products might have recognize that we used both the classical primitive and imprimitive actions of it in the proof of Proposition~\ref{Prop:Median}.

On one hand, since $H$ acts on $X$, it acts on $\powersetf{X}$.
Hence, for every $G$ action on $Y$ we have the imprimitive action of $G\wr_{\powersetf{X}} H$ on $Y\times \powersetf{X}$, which can be restricted to an action of the subgroup $G\wr_{X} H\leq G\wr_{\powersetf{X}} H$.

On the other hand, $G$ acts on itself by left multiplication.
It hence acts on the set $G'\coloneqq G\sqcup\{\varepsilon\}$ by fixing $\varepsilon$.
We hence have the primitive action of $G\wr_{X} H$ on $G'^X$.
Now, the set $\bigsqcup_XG$ naturally embeds as the subset of $G'^X$ consisting of all functions $\phi\colon X\to G'$ such that $\phi(x)=\varepsilon$ for all but one $x\in X$.
This subset is $G\wr_{X} H$ invariant, which gives us the desired action of $G\wr_{X} H$ on $\bigsqcup_XG$.
\end{rem}




\section{Related concepts and generalizations}\label{Section:Generalizations}
This short section is devoted to generalize some of the results of Section~\ref{Section:Median}, as well as to present some properties related to the FW property. Indeed, property FW does not stand alone and there are other similar properties that are of great interest.
\begin{defn}\label{Def:FHFA}
Let $G$ be a group.
It is said to have \emph{property SB} if any action on a metric space has bounded orbits.
It has \emph{property FH} if any action on a real Hilbert space has bounded orbits and \emph{property FA} if any action on a tree has bounded orbits.
Finally, $G$ is said to have (cofinality $\neq\omega$) if any action on an ultrametric space has bounded orbits.
\end{defn}
For countable groups (and more generally for $\sigma$-compact locally compact groups), property FH is equivalent to the celebrated Kazdhan's property (T) by the Delorme-Guichardet theorem, see for example \cite{MR2415834}, but this is not true in general \cite{MR2240370}.
\todo[inline]{Maybe expand a little more.}
The names FH, FW and FA come from the fact that this property admit a description in terms of existence of a Fixed point for action on Hilbert spaces, on spaces with Walls and on trees (\emph{Arbres} in french).
On the other hand, SB stands for Strongly Bounded and is sometimes called the Bergman property, while a group has cofinality $\neq\omega$ if and only if it cannot be written has an increasing union of proper subgroups.

We have the following strict implications \cite{MR1432323,MR0476875 ,MR3299841,2013arXiv1302.5982C}
\[
\textnormal{SB}\implies \textnormal{FH}\implies \textnormal{FW}\implies \textnormal{FA}\implies(\textnormal{cofinality}\neq\omega).
\]

It is possible to define other properties in the spirit of Definition~\ref{Def:FHFA}.
Let P be a property of metric spaces (for example \emph{be a connected median graph}) and BP be the group property: \emph{Every $G$-action on a space with P has bounded orbits}, where a $G$-action is supposed to ``preserve the P-structure''.
Lemmas \ref{Lemma:Subgroup} and \ref{Lemma:Semidirect} and Corollary \ref{Cor:Wreath} as well as their proofs remain true for groups with property BP.
Moreover, Lemma~\ref{Lemma:Sum}, Corollary~\ref{Cor:Sum}, as well as the part of Proposition~\ref{Prop:Median} saying that $G\wr_XH$ does not have $BP$ if $X$ is infinite, hold for properties that are stronger than property FW. 

On the other hand, the other results of Section~\ref{Section:Median} require a specific construction and do not generalize straightforward to groups with property BP.
Nevertheless, it is possible to extract the main ingredients of the proof and to adapt them to some specific cases.
We will now give a raw outline of this process, but let the details to the interested reader.

In order to generalize Example~\ref{Ex:MainMedian}, we will need to construct from a $G$ action on a set $X$ a $G$ action on a P-space $F(X)$,
such that $X$ embeds $G$-equivariently into $F(X)$.
%This will usually be done via a functor $F$ from the category of $G$-spaces to the category of $G$-$P$-spaces.
%More precisely, we say that it is \emph{possible to extend $G$ actions to P-spaces} if there is a $G$-equivariant map $i$ that associate to every $G$-set $X$ a P-space $Y$ endowed with a $G$ action and a $G$-equivariant map $\iota\colon X\hookrightarrow Y$.
For median graphs we took $F(X)=\powerset{X}$ in Example~\ref{Ex:MainMedian}, and then carefully choose some connected component of it, while for real Hilbert spaces it is possible to take $F(X)$ to be the real Hilbert space generated by $X$. That is, 
\[F(X)=\ell^2(X)=\bigg\{f\colon X\to \mathbf R\,\bigg|\, \sum_{x\in X}f(x)^2<\infty\bigg\},\]
where $G$ acts by permutation.
For property SB\todo{Rewrite ?}, we can take $F(X)$ to be  $\powerset{X}$ or $\ell^2(X)$, as well as many other possibilities.
In the context of trees, is is possible to take $F(X)=X\sqcup\{*\}$ with for every $x\in X$ an edge between $x$ and $*$ (the action of $G$ fixing $*$), while for ultrametric spaces it is always possible to take $F(X)=X$ with the discrete metric, however these last two examples will not be useful in practice.

In order to generalize Lemma~\ref{Lemma:Subgroup2}, we only need that a product of two P-spaces is still a P-space.
%More precisely, given two actions $G_i\curvearrowright P_i$, $i=1,2$, of groups on P-spaces we would like to have a natural action of $G_1\times G_2$ on some P-space $P_3$ containing $G_.
This property holds for metric spaces, ultra-metric spaces and also for Hilbert spaces where we take the direct sum $\mathcal H_1\oplus\mathcal H_2$.
On the other hand, for trees there is two natural candidats on which $G_1\times G_2$ acts: the cartesian product $T_1\times T_2$ or the tensor product $T_1\otimes T_2$, but none of them are tree ($T_1\times T_2$ has cycles while $T_1\otimes T_2$ is not connected).

Suppose that in the class of P-spaces the following hold: for any $G$-space $Y$ and any $H$-space $X$ we have a natural action of $G\wr_XH$ on $Y\times F(X)$. Then if $G\wr_X H$ has BP, so does $G$. This is the first part of Proposition~\ref{Prop:Median}.
We now explicit this for Hilbert spaces.
\begin{lem}
Let $\mathcal H$ be an Hilbert space with an isometric $G$-action and $X$ be a set with an $H$-action.
Then $G\wr_XH$ acts isometrically on the Hilbert space $\mathcal H\oplus \ell^2(X)$ by\todo{Check this}
\[
	\bigl(\phi,h\bigr).(y,f)=
	\begin{cases}
	(\phi(h.x).y,\delta_{h.x})&\textnormal{if $f=\delta_x$}\\
	(y,h.f)&\textnormal{if $f\neq\delta_x$.}
	\end{cases}
\]
In particular, if $G\curvearrowright \mathcal H$ has an unbounded orbit, so does $G\wr_XH\curvearrowright\mathcal H\oplus \ell^2(X)$.
\end{lem}

We hence recover\todo{Check si on peut enlever discret, + mettre réfs.}
\begin{prop}
Let $G$, $H$ be two discrete groups with $G$ non-trivial and $X$ a set on which $H$ acts. The wreath product $G \wr_X H$ has the property FH if and only if $G$ and $H$ have the property FH and $X$ is finite
\end{prop}
\todo[inline]{Regarder si on peut enlever groupe discret et ne pas passer par le résultat pour (T)}
%The proof is a direct application of the equivalence of the properties (FH) and (T) for discrete groups (see \cite{Bekka2008}) and of the Theorem \ref{T:Wreath_prop_T} .
%
It also follows from the above discussion that we have
\begin{prop}
Let $G$, $H$ be two discrete groups with $G$ non-trivial and $X$ a set on which $H$ acts. The wreath product $G \wr_X H$ has the property SB if and only if $G$ and $H$ have the property SB and $X$ is finite
\end{prop}
%
%
Finally, we mention the following result on property FA that was obtained by Cornulier and Kar
\begin{thm}[\cite{Cornulier2011}]
Let $G$, $H$ be two discrete groups with $G$ non-trivial and $X$ a set on which $H$ acts with a finite number of orbits and no fixed point. Then the wreath product $G \wr_X H$ has the property FA if and only if $H$ has the property FA and if $G$ is a group with a finite abelianization, which cannot be expressed as an union of proper increasing sequence of subgroups.
\end{thm}
\todo[inline]{regarder si c'est plus fort que $H$ possède la propriété (FA) et le lien avec le cardinal de $X$}

\todo[inline]{Regarder le cas PW, Haagerup,... i.e. action propre. Rappel: une action isométrique de $G$ est propre si pour tout $x$ (de manière équivalente il existe $x$), pour tout $r\in \mathbf R$ l'ensemble $\{g\in G\,|\,d(x,g.x)\}$ est fini.}
