%!TEX root = Property_FW_and_wreath_products.tex
\todo[inline]{Ajouter les petits lemmes sur les sous-groupes}
For finitely generated groups, the property FW can be expressed in terms of number of ends of Schreier graphs, see the point \ref{C:FW_Schrei} of the Proposition \ref{P:charact_FW}. In this section we will give a more constructive and explicit proof of the Proposition \ref{Prop:Median} for these kind of groups. 

We will begin by a short recall on Schreier graph. \todo{tile de redéfinir les Schreier ?}. There is a link between the actions of a group $G$ and its Schreier graph. Let $X$ be a $G$-set and $S$ a generating set of $G$. The graph of the action of $G$ on $X$ with respect to $S$ is the graph where the vertices are the elements of $X$ and two vertices are adjacent if we can go from one to the other using the action of an element on $S$. The graph of the action of $G$ on the orbit of an element $x$ of $X$ is isomorphic $\Sch(G,\stab(x),S)$, the Schreier graph associated to the stabilizer of $x$. In fact all the Schreier graph can be viewed as graph of action.

We will work with finitely generated group. The following lemma tells us when a wreath product is finitely generate.
\begin{lem}
Let $GH$ be groups and $X$ be a $H$-set. The wreath product $G \wr_X H$ is finitely generated if and only if $G$ and $H$ are finitely generated and if the number of orbits of the action $H \actsGroup X$ is finite.
\end{lem}
\begin{proof}
\todo[inline]{A faire}
\end{proof}

\begin{lem}\label{Lemma:Semidirect_ends}
Let $G = N \rtimes H$ be a finitely generated semidirect product. Then
\begin{enumerate}
\item If $G$ has FW, then so does $H$.
\item If both $N$ and $H$ have FW, then $G$ also has FW.
\end{enumerate}
\end{lem}
%
\begin{proof}
First of all we will recall that if $N$ is generated by a set $S$ and $H$ by a set $S'$, then $G$ is generated by $S \cup S'$. Let $\Gamma$ be a Schreier graph of $G$ with respect to $S \cup S'$.

Suppose that $H$ does not have the property FW, then there exists a Schreier graph $\Gamma$ of $H$ which have more than one end. The group $G$ acts on $H$ via 
\begin{equation*}
(n,h).x = hx 
\end{equation*}
for a vertex $x$ of $\Gamma$. This action $G \actsGroup \Gamma$ is transitive because the action $H \actsGroup \Gamma$ is. Then the graph $\Gamma'$ of the action $G \actsGroup \Gamma$ is exactly the graph $\Gamma$ with some additional loops coming from the generators on $S'$ and then $\Gamma'$ has more than one end.

If both $N$ and $H$ have FW, then each of their Schreier graphs have at most one end. The groups $N$ and $H$ acts on $\Gamma$ via
\begin{align*}
n.x &= (n,0)x \\
h.x &= (0,h)x
\end{align*}
for a vertex $x$ of $\Gamma$. For each such a vertex $x$ we define $\Gamma_x^H$ (and respectively $\Gamma_x^N$) the Schreier graph obtained from the action of $N$ (resp. $H$) on the orbit of $x$. This is a subgraph of $\Gamma$. As $N$ and $H$ have FW, then $\Gamma_x^H$ and $\Gamma_x^N$ are finite or one-ended. We want to prove that every Schreier graph has at most one end. Let $\Gamma$ be an infinite Schreier graph of $G$ and $K$ be a finite subset of vertices. We will construct a path between every pair of vertices which are in infinite connected component.

First, for every $x$ in $K$, if $\Gamma_x^H$ is finite we add all the vertices of this subgraph in $K$. Adding vertices in $K$ leave the number of ends unchanged. Then for every $x$ in $\Gamma \setminus K$ we are sure that $\Gamma_x^H \setminus K$ is equal to $\Gamma_x^H$ or has one end. 

Let $x$ and $y$ be two vertices of a infinite component of $\Gamma \setminus K$. The action of $G$ on $\Gamma$ is transitive, then there exists an element $(n,h)$ such that $(n,h).x = y$. We choose a $z$ in $\Gamma_x^H$ such that $\Gamma_z^N$ does not contain an element of $K$ or is one-ended and such that $z'=(h'h^{-1}.n,0)z$ is not in $K$, where $h'$ is in $H$ and $h'.x = z$. Such an element always exists as $\Gamma_x^H$ and $\Gamma_y^H$ are infinite and $K$ is finite. The vertex $z'$ is in $\Gamma_y^H$ because
\begin{equation*}
(0,hh'^{-1}) z' = (0,hh'^{-1})(h'h^{-1}.n,0)(0,h')x = (n,h)x = y.
\end{equation*}
%

We will construct a path on $\Gamma \setminus K$ between $x$ and $y$ as follows. The subgraph $\Gamma_x^H$ is one-ended, then there exists a path between $x$ and $z$. In the same way, $\Gamma_z^N$ has one end or has no vertex in $K$, then there is a path which join $z$ and $z'$. There exists a path between $z'$ and $y$ in $\Gamma_y^H$ which is one-ended. Then $x$ and $y$ are path-connected and then $\Gamma$ has one end. 
\end{proof}
%
%
\begin{cor}\label{Cor:Wreath_ends}
Let $G\wr_X H$ be a finitely generated wreath product of $G$ and $H\curvearrowright X$.
Then
\begin{enumerate}
\item
If $G\wr_X H$ has FW, then so does $H$.
\item
If $G$ and $H$ have FW and $X$ is finite, then $G\wr_X H$ has FW.
\end{enumerate}
\end{cor}
%
%
\begin{lem}\label{Lem:Wreath_groups_ends}
Let $G \wr_X H$ be a finitely generated wreath product and $G$ be a non-trivial group. If $G \wr_X H$ has the property FW, then 
\begin{enumerate}
\item $G$ has the property $FW$.
\item $H$ has the property $FW$.
\item $X$ is finite.
\end{enumerate}
%
\begin{proof} \todo{Rendre cohérente la notation et l'ordre des preuves}
Let us begin by fixing the notation. We denote by $S$ (and respectively by $S'$) a finite generating set of $G$ (and respectively of $H$). We choose an arbitrary point of $x_0$ of $X$. The group $\Gamma = G \wr_X H $ is generated by the set 
\begin{equation*}
\S = \{(\delta_{x_0}^s, e_h) : s \in S \} \cup \{ (0, s') : s' \in S' \} 
\end{equation*}
where 
\begin{equation*}
\delta_{x_0}^s = \begin{cases} e_g & x \neq x_0 \\ s 	& x = x_0 \end{cases} \quad \text{and}\quad 0(x) = e_g \quad \forall x \in X.
\end{equation*}
We will prove the contrapositives. The idea is to construct, for each of the three cases, a Schreier graph of $\Gamma$ with more than one end. To do this we will consider actions of $\Gamma$ and associated graph's action. We will treat the 3 cases separately.

Suppose that $X$ is an infinite set. We define $Y = G \times X$ and an action of $\Gamma$ on $Y$ as
\begin{equation*}
(\phi,h) \cdot (g,x) = (\phi(hx)g, hx)
\end{equation*}
for $(\phi,h)$ in $\Gamma$ and $(g,x)$ in $Y$. This action is transitive. Indeed, let $(g_1,x_1)$ and $(g_2,x_2)$ be two elements of $Y$. By transitivity of the action of $H$ on $Y$, there exists $h$ in $H$ such that $hx_1 = x_2$. We can always find $\phi$ in $\bigoplus_X G$ such that $\phi(hx_1) = g_2g_1^{-1}$. Then, 
\begin{equation*}
(\phi,h) (g_1,x_1) = (\phi(h x_1) g, h x_1) = (g_2,x_2).
\end{equation*}
The graph of the action of $\Gamma$ on $Y$ is isomorphic to the Schreier graph $\Sch(\Gamma, \stab(e_G,x_0), \S)$. We decompose the graph into leaves of the form $Y_g = \{ g \} \times G$. There are two types of edges on this graph which are coming from the two types of generators. The first one, of the form $(0,s')$, give us on each leaf a copy of the graph of the action of $H$ on $X$. Indeed, we have
\begin{equation*}
(0,s')(g,x) = (g, s'x).
\end{equation*}
Tthe second one, of the form $(\delta_{x_0}^s,0)$, give us loops everywhere excepting on vertices with $x_0$ as second coordinate. By direct computation, we see that a leaf $Y_g$ is connected to $Y_{sg}$ by the vertices $(g,x_0)$ and $(sg,x_0)$: 
\begin{equation*}
(\delta_{x_0}^s,0)(g,x) = 
\begin{cases}
(g,x) & x \neq x_0 \\
(sg, x) & x = x_0
\end{cases}
\end{equation*}
\missingfigure{Structure du Schreier et des feuilles}.

If we remove a vertex $(g,x_0)$ we disconnect the leaf $Y_g$ to the graph. As $X$ is infinite, the number of ends is strictly greater than 1.

Suppose that $G$ does not have the property FW. By the point \ref{C:FW_Schrei} of the Proposition \ref{P:charact_FW} there exists a subgroup $K$ of $G$ such that $\Sch(G,K,S)$ has more than one end. The group $\Gamma$ acts on $G/K \times X$:  
\begin{equation*}
(\phi,h)(gK,x) = (\phi(hx) gK, hx).
\end{equation*}
As above, the action is transitive and the graph of this action is isomorphic to a Schreier graph. We decompose this graph into leaves in the same way. Now we look at the subgraph made up of vertices $(g,x_0)$ and edges $(\delta_{x_0}^s,0)$ and we remark that it is isomorphic the Schreier graph $\Sch(G,K,S)$ which has more than one end. Then our graph has also more than one end.
\missingfigure{Même image mais avec les sous-Schreier en évidence}.

Suppose that $H$ does not have the property FW. There is a subgroup $K$ of $H$ such that $\Sch(H,K,S')$ has more than one end. We consider the action of $\Gamma$ on $H/K$ defined as
\begin{equation*}
 (\phi,h)h'K = hh'K.
\end{equation*}
This is a transitive action. All the edges of type $(\delta_{x_0}^s, 0)$ are loops and the edges $(0, s')$ are the edges of $\Sch(H,K,S')$ which has more than one end. Then this Schreier graph has also more than one end.
\end{proof}
\end{lem}
%
%
The following proposition is a direct application of Corollary \ref{Cor:Wreath_ends} and Lemma \ref{Lem:Wreath_groups_ends}.
\begin{prop}
Let $G$ be a non trivial finitely generated group, $H$ be a finitely generated group and $X$ a set on which $H$ acts with a finite number of orbit. The wreath product $G \wr_X H$ has the property FW if and only if $G$ and $H$ have the property FW and $X$ is finite.
\end{prop}
%